% - Nociones de filtro, ultrafiltro, ideal y filtros k-completos.
% - Conjuntos cerrados no acotados y estacionarios.
% - Teoría de modelos, conceptos básicos.
% - Inmersiones elementales y puntos críticos, haciendo mención de cardinales medibles.
\chapter{Nociones básicas}

Este capítulo establece varios conceptos básicos que serán necesarios
más adelante. Las nociones de filtro, ideal, ultrafiltro y filtro $\kappa$-completo
junto con los conjuntos no acotados y estacionarios componen las definiciones de
conjuntos más elementales que harán falta.

Luego, un rápido repaso de la teoría de modelos permitirá abordar las inmersiones
elementales, que son una pieza central del teorema de Kunen.

\section{Filtros}

Esta sección se ocupa de dar las definiciones básicas de filtros,
que serán necesarias a lo largo del texto.
Los filtros caracterizan a conjuntos \say{grandes} dentro
de un conjunto dado $C$.

\begin{defi}
    Sea $C$ un conjunto no vacío. Un conjunto $F\subset \pwset{C}$ es un
    \concept{filtro} si se cumplen las siguientes condiciones:
    \begin{enumerate}[label=\alph*)]
        \item $C\in F$ y $\emptyset\notin F$.
        \item Si $X,Y\in F$ entonces $X\cap Y\in F$.
        \item Si $X,Y\subset C$, $X\in F$ y $X\subset Y$ entonces $Y\in F$.
    \end{enumerate}
\end{defi}

\begin{defi}
    Sea $F$ un filtro sobre $C$. $F$ es \concept{ultrafiltro} si, para todo $X\subset C$,
    se tiene que $X\in F$ o $X-S\in F$.
\end{defi}

Una caracterización equivalente para ultrafiltros viene dada por la propiedad
de maximalidad:

\begin{teo}
    Sea $F$ un filtro sobre $C$. $F$ es \concept{ultrafiltro} si, y solo si, es maximal.
\end{teo}

La siguiente definición es central para la teoría de cardinales medibles.

\begin{defi}
    Sea $\kappa$ un cardinal regular y $F$ un filtro sobre $C$.
    $F$ es $\kappa$-completo siempre que dada una familia de conjuntos
    $\set{X_\alpha\in F\colon \alpha<\kappa}$,
    se tiene que
    \[
        \bigcap X_\alpha \in F.
    \]
\end{defi}

Un ejemplo que une los conceptos tratados hasta ahora es, como ya se mencionó,
la definición de cardinal medible.

\begin{defi}
    Sea $\kappa > \omega$ un cardinal. $\kappa$ es \concept{medible} si existe
    un ultrafiltro $\kappa\text{-completo}$ sobre $\kappa$.
\end{defi}

\section{Conjuntos Estacionarios}

El principal objetivo de esta sección es establecer un teorema
de Solovay, acerca de particiones
con conjuntos estacionarios, usando el \cref{teo:fodor}
de Fodor.

Sea $C$ un conjunto y $X\subset C$, diremos que $X$ es \concept{no acotado}
en $C$ si $\sup(X) = C$.
Si $C$ es además un conjunto de ordinales, un ordinal límite $\alpha$ es
\concept{punto límite} de $C$ si $\sup ( C \cap\alpha ) = \alpha$.
\begin{defi}
    Sea $\kappa$ un cardinal regular no numerable. Un conjunto $C\subset \kappa$
    es \concept{cerrado no acotado} (\cna) si $C$ es no acotado en $\kappa$ y contiene a
    todos sus puntos límites menores que $\kappa$.
    Un conjunto $S\subset\kappa$ es \concept{estacionario} si para cada conjunto
    \cna{} $C\subset\kappa$ se tiene $S\cap C\neq\emptyset$.
\end{defi}

Será de utilidad saber el comportamiento de los conjuntos \cna{} bajo intersecciones.
Para este fin, definimos, dada $\op{X_\alpha\colon \alpha<\kappa}$ una sucesión
de subconjuntos de $\kappa$, la \concept{intersección diagonal} de
$X_\alpha$ como:
\[
    \dint_{\alpha<\kappa}X_\alpha
    =
    \set{\epsilon<\kappa\colon\epsilon\in \bigcap_{\alpha<\epsilon}X_\alpha} .
\]

\begin{teo}\label{teo:intersection-cna}
    Sea $\kappa$ un cardinal regular no numerable y $\set{C_\alpha}_{\alpha<\kappa}$ una familia
    de \cna{} en $\kappa$, entonces:
    \begin{enumerate}[label=\alph*)]
        \item $C_\alpha\cap C_\beta$ es \cna{} ($\alpha,\beta < \kappa$).
        \item $\bigcap_{\alpha<\kappa}C_\alpha$ es \cna.
        \item $\dint_{\alpha<\kappa}C_\alpha$ es \cna.
    \end{enumerate}
\end{teo}

\begin{proof}
    Veamos cada parte por separado.

    \begin{enumerate}[label=\alph*)]
        \item\label{pr:intersection-simple-cna}
            Es claro que $C\cap D$ es cerrado. Veamos que es no acotado.
            Sea $\alpha<\kappa$. Dado que $C$ es no acotado, existe $\alpha_1\in C$
            tal que $\alpha_1 > \alpha$. De la misma forma, existe $\alpha_2\in D$
            tal que $\alpha_2 > \alpha_1$. Podemos seguir con este proceso para obtener
            una sucesión creciente:
            \[
                \alpha < \alpha_1 < \alpha_2 < \dots
            \]
            Sea $\beta$ el límite de la sucesión de arriba.
            Entonces $\beta < \kappa$ y $\beta\in C$ y $\beta\in D$.


        \item\label{pr:intersection-cna}
            La demostración será por inducción.
            Sea $\lambda<\kappa$ y $\seq{C_\alpha\colon\alpha<\lambda}$
            una sucesión de conjuntos \cna{} en $\kappa$.
            Para los ordinales sucesores, podemos simplemente aplicar
            el \cref{pr:intersection-simple-cna}.
            Si $\lambda$ es ordinal límite, asumiremos que el teorema
            es cierto para cada $\alpha<\lambda$. Podemos ahora sustituir
            cada $C_\alpha$ por $\bigcap_{\xi\leq\alpha} C_\xi$ y obtenemos
            una sucesión decreciente con la misma intersección. Entonces a partir de ahora:
            \[
                C_0 \subset C_1 \subset C_2 \subset \dots
            \]
            serán \cna{} y $C = \bigcap_{\alpha<\lambda} C_\alpha$.
            Por la misma razón que el \cref{pr:intersection-simple-cna}, no es difícil
            ver que $C$ es cerrado. Veamos que es no acotado. Sea $\alpha<\kappa$,
            construiremos una sucesión de la siguiente forma: sea $\beta_0\in C_0$ mayor que
            $\alpha$, y para cada $\xi<\gamma$ se tomará $\beta_\xi\in C_\xi$
            tal que $\beta_xi > \sup\set{\beta_\nu\colon\nu<\xi}$.
            Dado que $\kappa$ es regular y $\gamma<\kappa$, la sucesión que se acaba de
            describir existe y su límite $\beta$ es menor que $\kappa$.
            Para cada $\eta<\gamma$, $\beta$ es límite de una sucesión
            $\seq{\beta_xi\colon\eta\leq\xi<\gamma}$ en $C_\eta$, por lo que
            $\beta\in C_\eta$ y esto implica $\beta\in C$.


        \item Llamemos $D$ a $\dint_{\alpha<\kappa}C_\alpha$. Veamos primero que $D$ es cerrado.
            Sea entonces $\lambda<\kappa$ tal que $D\cap\lambda$
            no está acotado en $\lambda$, esto es, que $\lambda$ es punto límite de $D$.
            Tomemos $\beta\in\lambda$,
            entonces existe $\epsilon\in\lambda\cap D$ tal que $\beta<\epsilon$ pues
            $D\cap\lambda$ es no acotado.
            Como $\epsilon\in D$, existe $C_\alpha$, con $\alpha<\epsilon<\lambda$,
            al que $\epsilon$ pertenece.
            Pero entonces, lo que hemos demostrado es que siempre que tomemos $\beta\in\lambda$
            existe $\epsilon\in C_\alpha\cap\lambda$ que esta por encima de $\beta$ o, equivalentemente,
            que $C_\alpha\cap\lambda$ es no acotado en $\lambda$.
            Al ser $C_\alpha$ cerrado tenemos $\lambda\in C_\alpha$ y esto implica
            $\lambda\in D$. Luego $D$ es cerrado.

            Solo falta ver que $D$ es no acotado en $\kappa$.
            Para esto notemos que, debido al \cref{pr:intersection-cna},
            se puede reemplazar cada $C_\alpha$ por $\bigcap_{\xi\leq\alpha} C_\xi$
            y obtenemos una sucesión decreciente $C_0\subset C_1\subset\dots$
            que no cambia el valor de $D$.
            Sea $\gamma\in\kappa$. Como cada $C_\alpha$ es no acotado en $\kappa$,
            podemos construir una sucesión $\seq{\beta_n\colon n\in\omega}$ de la siguiente forma:
            tomamos $\beta_0\in C_0$ mayor que $\gamma$, luego dado $\beta_n$, tomamos
            $\beta_{n+1} \in C_{\beta_n}$ mayor que $\beta_n$. Llamemos $\beta = \lim_n\beta_n$
            y tomemos $\xi<\beta$. Entonces existe $\beta_n>\xi$ y cada $\beta_k$ con $k>n$
            pertenece a $C_{\beta_n}$, pues los $C_\alpha$ están encajados,
            por lo que $\beta\in C_{\beta_n}$ y $\beta\in C_\xi$.
            Pero esto muestra que $\beta\in D$ y que $D$ es no acotado.
    % Queremos ver que $\alpha\in D$ y
    % esto es equivale a que existe $\alpha<\lambda$ tal que $\lambda\in C_\alpha$.
    \end{enumerate}
\end{proof}

El siguiente resultado se sigue del \cref{teo:intersection-cna}.

\begin{cor}\label{cor:cna-stationary}
    Sea $C$ un conjunto \cna{} en $\kappa$ y $E$ estacionario, también en $\kappa$.
    Entonces $C\cap E$ es estacionario.
\end{cor}

\begin{defi}
    Una función de ordinales $f$ en un conjunto $S$ es \concept{regresiva}, si
    $f(\alpha)<\alpha$ para todo $\alpha\in S$.
\end{defi}

\begin{teo}[Fodor]\label{teo:fodor}
    Sea $f$ una función regresiva en un conjunto estacionario $E\subset\kappa$.
    Entonces existe $\alpha\in\kappa$ tal que $f^{-1}(\set{\alpha})$ es estacionario.
\end{teo}

\begin{proof}
    Supongamos, en busca de una contradicción, que $f^{-1}(\set{\alpha})$ no es estacionario
    para todo $\alpha<\kappa$. Entonces existen conjuntos \cna{} $C_\alpha$ tales que
    $C_\alpha\cap f^{-1}(\set{\alpha}) = \emptyset$. Si $D=\dint_{\alpha<\kappa} C_\alpha$,
    entonces por el \cref{teo:intersection-cna}, $D$ es \cna{} en $\kappa$.
    Pero entonces $D\cap E$ es estacionario (debido al \cref{cor:cna-stationary})
    y podemos tomar $\gamma\in D\cap E$.
    Como $\gamma\in E$, sea $\delta=f(\gamma)<\gamma$.
    Entonces $\gamma\in f^{-1}(\set{\delta})\cap C_\delta$, lo cual es una contradicción.
\end{proof}

El siguiente es un teorema auxiliar, que será de utilidad para el \cref{teo:solovay}.

\begin{teo}\label{teo:stationary}
    Sea $E\subset\kappa$ un conjunto estacionario en $\kappa$. Entonces el conjunto
    \[
        T=\set{\lambda\in E\colon
            \cf\lambda = \aleph_0 \lor
            S\cap\alpha\;\text{no es un subconjunto estacionario de $\alpha$}}
    \]
    es estacionario en $\kappa$.
\end{teo}

\begin{teo}[Solovay]\label{teo:solovay}
    Sea $\kappa$ un cardinal regular no numerable. Entonces cada subconjunto estacionario
    de $\kappa$ es la unión disjunta de $\kappa$ subconjuntos estacionarios.
\end{teo}

\begin{proof}
    Sea $A$ un subconjunto estacionario de $\kappa$.
    Por el \cref{teo:stationary}, asumiremos que el conjunto $W$
    consistente de todos los $\alpha\in A$ tales que $\alpha$ es cardinal
    regular y $A\cap\alpha$ no es estacionario en $\alpha$, es estacionario en $\kappa$.
    Existe entonces un conjunto \cna{} $C_\alpha\subset\alpha$ tal que $A\cap C_\alpha = \emptyset$.
    Notemos que, por definición, $W\subset A$ por lo que $C_\alpha\cap W=\emptyset$.
    Sea $\seq{a_\xi^\alpha\colon\xi<\alpha}$ la enumeración creciente de $C_\alpha$,
\end{proof}

\section{Teoría de Modelos}
\label{sec:models}

La teoría de modelos es un área relativamente joven \autocite[pág. 3]{chang_model_2012}.
No obstante, su desarrollo ha sido crucial para la teoría de conjuntos y los
cardinales grandes \autocite[pág. xv]{kanamori_higher_2009}.

Se quiere definir lo que es un modelo para un lenguaje formal $\lex{L}$.
Un lenguaje $\lex{L}$ es un conjunto de símbolos relacionales, funcionales y constantes.
Los símbolos relacionales y funcionales pueden tener cualquier cantidad finita de argumentos,
lo que se conoce usualmente como su aridad, excepto cero.

Dado un conjunto cualquiera $A$, interesa darle significado a los símbolos de un
lenguaje $\lex{L}$ en $A$. Esto se logra a través de una \concept{interpretación}, esto es,
una correspondencia que asigna a cada relación $n$-aria $P$ una relación
$R\subset A^n$, a cada función $m$-aria una función $G\colon A^m\to A$ y a cada
constante $c$ un elemento $x\in A$.

\begin{defi}\label{def:model}
    Sea $\lex{L}$ un lenguaje formal. Un \concept{modelo} $\model{A}$ para $\lex{L}$ se define como,
    \[
        \model{A} = \op{A, \mathcal{I}}.
    \]
    Donde $A$, que es un conjunto cualquiera, es el \concept{universo} de $\model{A}$ y
    $\mathcal{I}$ es una interpretación de los símbolos de $\lex{L}$ en $A$.
\end{defi}

Dada una sentencia $\phi$ de un lenguaje $\lex{L}$ y $\model{A}$ un modelo para $\lex{L}$,
se escribirá $\model{A}\models\phi$ si la fórmula $\phi$ se satisface en $\model{A}$.
Intuitivamente, la relación $\models$ quiere decir que $\phi$ es verdadera en el modelo.
Una definición rigurosa de $\models$ es posible, y requiere inducción sobre la complejidad
de $\phi$ (véase \autocite[\S 1.3]{chang_model_2012} ó \autocite[\S 12]{jech_set_2003}).

Dados dos modelos $\model{A}, \model{B}$ se dirá que $\model{A}$ es \concept{elementalmente
    equivalente} a $\model{B}$, en símbolos $\model{A}\equiv \model{B}$, si toda sentencia
que es verdadera en $\model{A}$ lo es también en $\model{B}$ y viceversa.

\note{Explicar un poco más que es $\lex{L}_\in$ y los $\in$-modelos}
La \cref{def:model} esta dada en forma general. Normalmente interesarán modelos de $\lex{L}_\in$,
el lenguaje de la teoría de conjuntos, o $\in$-modelos de la forma $\op{A,\in}$.

\section{Inmersiones Elementales}
\label{sec:elem-embed}
