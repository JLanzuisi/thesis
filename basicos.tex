% - Nociones de filtro, ultrafiltro, ideal y filtros k-completos.
% - Conjuntos cerrados no acotados y estacionarios.
% - Teoría de modelos, conceptos básicos.
% - Inmersiones elementales y puntos críticos, haciendo mención de cardinales medibles.
\chapter{Nociones básicas}

Este capítulo establece varios conceptos básicos que serán necesarios
más adelante. Las nociones de filtro, ideal, ultrafiltro y filtro $\kappa$-completo
junto con los conjuntos no acotados y estacionarios componen las definiciones de
conjuntos más elementales que harán falta.

Luego, un rápido repaso de la teoría de modelos permitirá abordar las inmersiones
elementales, que son una pieza central del teorema de Kunen.

\section{Filtros}

Los filtros caracterizan a conjuntos \say{grandes} dentro
de un conjunto dado $C$.
\begin{defi}
    Sea $C$ un conjunto no vacío. Un conjunto $F\subset \pwset{C}$ es un
    \concept{filtro} si se cumplen las siguientes condiciones:
    \begin{enumerate}[label=\alph*)]
        \item $C\in F$ y $\emptyset\notin F$.
        \item Si $X,Y\in F$ entonces $X\cap Y\in F$.
        \item Si $X,Y\subset C$, $X\in F$ y $X\subset Y$ entonces $Y\in F$.
    \end{enumerate}
\end{defi}

Dado un filtro cualquiera $F$ es natural considerar el filtro más grande
que contiene a $F$. Si dicho filtro no existe, $F$ es maximal y le llamamos ultrafiltro.
Dicho de otra manera:

\begin{defi}
    Sea $F$ un filtro sobre $C$. $F$ es \concept{ultrafiltro} si, y solo si, es maximal.
\end{defi}

La siguiente definición es central para la teoría de cardinales medibles.

\begin{defi}
    Sea $\kappa$ un cardinal regular y $F$ un filtro sobre $C$.
    $F$ es $\kappa$-completo siempre que dada una familia de conjuntos
    $\set{X_\alpha\in F\colon \alpha<\kappa}$,
    se tiene que
    \[
        \bigcap X_\alpha \in F.
    \]
\end{defi}

Un ejemplo que une los conceptos tratados hasta ahora es, como ya se mencionó,
la definición de cardinal medible.

\begin{defi}
    Sea $\kappa > \omega$ un cardinal. $\kappa$ es \concept{medible} si existe
    un ultrafiltro $\kappa\text{-completo}$ sobre $\kappa$.
\end{defi}

% \begin{defi}
%     Sea $F$ un filtro sobre un cardinal $\kappa$. $F$ es \concept{normal}
%     si, y solo si, es cerrado bajo intersecciónes diagonales:
%     \[
%         \text
%         {
%             Si $\op{X_\alpha\mid \alpha<\kappa}$ es una sucesión sobre $F$,
%             $\dint_{\alpha<\kappa} X_\alpha \in F$.
%         }
%     \]
% \end{defi}

% \section{Ideales}

% Un \concept{ideal} es el concepto dual de un filtro, en el sentido siguiente:
% si $F$ es un filtro sobre $C$ entonces el conjunto
% $I=\set{ C - X \colon X\in F }$
% es un ideal. Los ideales caracterizan conjuntos ``pequeños''.
% \begin{defi}
%     Sea $C$ un conjunto no vacío. Un conjunto $I\subset \pwset{C}$ es un
%     \concept{ideal} si se cumplen las siguientes condiciones:
%     \begin{enumerate}[label=\alph*)]
%         \item $\emptyset\in I$ y $C\notin I$.
%         \item Si $X,Y\in I$ entonces $X\cup Y\in I$.
%         \item Si $X,Y\subset C$, $X\in I$ y $Y\subset X$ entonces $Y\in I$.
%     \end{enumerate}
% \end{defi}


\section{Conjuntos Estacionarios}

Sea $C$ un conjunto y $X\subset C$, diremos que $X$ es \concept{no acotado}
en $C$ si $\sup(X) = C$.
Si $C$ es además un conjunto de ordinales, un ordinal límite $\alpha$ es
\concept{punto límite} de $C$ si $\sup ( C \cap\alpha ) = \alpha$.
\begin{defi}
    Sea $\kappa$ un cardinal regular no numerable. Un conjunto $C\subset \kappa$
    es \concept{cerrado no acotado} si $C$ es no acotado en $\kappa$ y contiene a
    todos sus puntos límites menores que $\kappa$.
    Un conjunto $S\subset\kappa$ es \concept{estacionario} si para cada conjunto
    cerrado no acotado $C$ se tiene $S\cap C\neq\emptyset$.
\end{defi}

Sea $\op{X_\alpha\colon \alpha<\kappa}$ una sucesión
de subconjuntos de $\kappa$. La \concept{intersección diagonal} de
$X_\alpha$ se define como:
\[
    \dint_{\alpha<\kappa}X_\alpha
    =
    \set{\epsilon<\kappa\colon\epsilon\in \bigcap_{\alpha<\epsilon}X_\alpha} .
\]

\section{Teoría de Modelos}
\label{sec:models}

La teoría de modelos es un área relativamente joven \autocite[pág. 3]{chang_model_2012}.
No obstante, su desarrollo ha sido crucial para la teoría de conjuntos y los
cardinales grandes \autocite[pág. xv]{kanamori_higher_2009}.

Se quiere definir lo que es un modelo para un lenguaje formal $\lex{L}$.
Un lenguaje $\lex{L}$ es un conjunto de símbolos relacionales, funcionales y constantes.
Los símbolos relacionales y funcionales pueden tener cualquier cantidad finita de argumentos,
lo que se conoce usualmente como su aridad, excepto cero.

Dado un conjunto cualquiera $A$, interesa darle significado a los símbolos de un
lenguaje $\lex{L}$ en $A$. Esto se logra a través de una \concept{interpretación}, esto es,
una correspondencia que asigna a cada relación $n$-aria $P$ una relación
$R\subset A^n$, a cada función $m$-aria una función $G\colon A^m\to A$ y a cada
constante $c$ un elemento $x\in A$.

\begin{defi}\label{def:model}
    Sea $\lex{L}$ un lenguaje formal. Un \concept{modelo} $\model{A}$ para $\lex{L}$ se define como,
    \[
        \model{A} = \op{A, \mathcal{I}}.
    \]
    Donde $A$, que es un conjunto cualquiera, es el \concept{universo} de $\model{A}$ y
    $\mathcal{I}$ es una interpretación de los símbolos de $\lex{L}$ en $A$.
\end{defi}

Dada una sentencia $\phi$ de un lenguaje $\lex{L}$ y $\model{A}$ un modelo para $\lex{L}$,
se escribirá $\model{A}\models\phi$ si la fórmula $\phi$ se satisface en $\model{A}$.
Intuitivamente, la relación $\models$ quiere decir que $\phi$ es verdadera en el modelo.
Una definición rigurosa de $\models$ es posible, y requiere inducción sobre la complejidad
de $\phi$ (véase \autocite[\S 1.3]{chang_model_2012} ó \autocite[\S 12]{jech_set_2003}).

Dados dos modelos $\model{A}, \model{B}$ se dirá que $\model{A}$ es \concept{elementalmente
    equivalente} a $\model{B}$, en símbolos $\model{A}\equiv \model{B}$, si toda sentencia
que es verdadera en $\model{A}$ lo es también en $\model{B}$ y viceversa.

\note{Explicar un poco más que es $\lex{L}_\in$ y los $\in$-modelos}
La definición \ref{def:model} esta dada en forma general. Normalmente interesarán modelos de $\lex{L}_\in$,
el lenguaje de la teoría de conjuntos, o $\in$-modelos de la forma $\op{A,\in}$.

\section{Inmersiones Elementales}
\label{sec:elem-embed}
