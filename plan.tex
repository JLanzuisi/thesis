\documentclass[12pt]{article}

\usepackage{fontsetup}
\usepackage{mathtools}

\begin{document}

\textsc{Dependencias para las 3 demostraciones}

\begin{enumerate}
  \item La demostración original de Kunen parece bastante auto-contenida,
    solo necesita de la noción de función $\omega$-Jónsson que se da en ese
    mismo capítulo unos párrafos mas arriba.
  \item Para la demostración de Woodin hace falta el teorema \S16.9 de Solovay sobre
    particiones y conjuntos estacionarios.
    \begin{enumerate}
      \item Para \S16.9 hace falta hablar de ideales normales $\kappa$ saturados.
      \item dar la definición de ideales $\kappa$ saturados.
      \item luego la definición de ideales \emph{normales} $\kappa$ saturados.
        Para esto hace falta la definición de filtro normal.
      \item demostrar \S16.8(d), que depende de las partes anteriores,
        así que hay que demostrar \S16.8(a)-(d).
    \end{enumerate}
  \item Para la demostración de Harada hace falta \S22.4(b), que habla de
    ultrafiltros $\omega_1$-completos.
    \begin{enumerate}
      \item Tendría que demostrar \S22.4(a), pues la parte (b) se sigue de esta.
      \item Para \S22.4(a) es bueno haber demostrado \S5.7(d)
    \end{enumerate}
\end{enumerate}

\textsc{Como incluirlas en la tesis}

\begin{enumerate}
  \item Dar la definición de filtro normal cuando se habla de filtros.
  \item También en la parte de filtros, hablar de ultrafiltros $\omega_1$-completos.
  \item Definir ideales saturados cuando se habla de ideales.
  \item Luego definir ideales normales saturados en la misma sección sobre ideales.
  \item Justo después de ideales normales saturados se puede demostrar \S16.8(a)-(d).
  \item Demostrar \S16.9 en el capítulo de ideales y filtros, después de definir
    conjuntos estacionarios.
  \item Demostrar \S22.4(a) en capítulo sobre inmersiones elementales,
    después de \S5.7(d) que debería ser el resultado anterior.
\end{enumerate}

\end{document}
