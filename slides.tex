\documentclass{beamer}

\RequirePackage { geometry }
\RequirePackage { setspace }
\RequirePackage { microtype }
\RequirePackage { enumitem }
\RequirePackage [ spanish ] { babel }
\RequirePackage { csquotes }
\RequirePackage { biblatex }
\RequirePackage { xcolor }
\RequirePackage { hyperref }
\RequirePackage { graphicx }
\RequirePackage { mathtools, amsthm, amsfonts }
\RequirePackage { dirtytalk }
\RequirePackage [ spanish ] { cleveref }
\RequirePackage { epigraph }

\ExplSyntaxOn
\makeatletter

% Cref
\crefname{section}{\S}{\S}

% Set page margins
\geometry
    {
        includehead,
        paper = letterpaper,
        left = 3cm,
        right = 2cm,
        bottom = 2cm,
        top = 2cm,
        marginparsep = 10pt,
        marginparwidth = 2cm,
    }

% Set fonts
\usepackage[lf]{Baskervaldx} % lining figures
\usepackage[bigdelims,vvarbb,baskervaldx]{newtxmath} % math italic letters from Nimbus Roman
\usepackage[cal=boondoxo]{mathalfa} % mathcal from STIX, unslanted a bit
\usepackage[scale=.9]{courierten}
\newcommand{\titlefont}{\bfseries}

% Set basic spacing
\setlength { \parindent } { 8.46pt }
\setlength { \parskip } { 6pt }
\onehalfspacing

% Headers and footer
\pagestyle{headings}
\gdef\@oddhead{\hfil\thepage}

% Bibliography
\addbibresource{references.bib}

% Hyperref
\hypersetup
    {
        colorlinks = true,
        allcolors = black,
    }

% Theorems
\newtheoremstyle{defi}% hnamei
    {.5\baselineskip}% hSpace abovei
    {.5\baselineskip}% hSpace belowi
    {}% hBody fonti
    {}% hIndent amounti
    {\scshape}%\addfontfeatures{LetterSpace=5}}% hTheorem head fonti
    {.}% hPunctuation after theorem headi
    {.5em}% hSpace after theorem headi
    {}% hTheorem head spec (can be left empty, meaning ‘normal’)i
\theoremstyle{defi}
\newtheorem{defi}{Definición}[chapter]

% Sectioning commands
\usepackage[explicit]{titlesec}
\titleformat
    {\chapter}
    [display]
    {\titlefont}
    {
        \centering
        \fontsize{14}{14}\selectfont
        \UpperCase{Capítulo}~\thechapter\\[10pt]
        \fontsize{12}{12}\selectfont
        \UpperCase{#1}
    }
    {0ex}
    {\centering}
\titleformat{name=\chapter, numberless}
    [block]
    {\titlefont}
    {\fontsize{14}{14}\selectfont\UpperCase{#1}}
    {0ex}
    {\centering}
\titleformat
    {\section}
    {\titlefont}
    {\thesection\fontsize{12}{12}\selectfont\ #1}
    {0ex}
    {}
\titleformat{name=\section, numberless}
    [block]
    {\titlefont}
    {\fontsize{12}{12}\selectfont #1}
    {0ex}
    {}

% USB logo
\cs_set_eq:NN \latex_centering:D \centering

\cs_new:Npn \th_print_logo_head:n #1
    {
        \group_begin:
            \latex_centering:D
            \includegraphics[scale=.3]{resources/usblogo} \\
            { \titlefont \UpperCase{#1} }
            \tex_par:D
        \group_end:
    }

\NewDocumentCommand{ \UpperCase }{ m }
    {
        \group_begin:
        %\addfontfeatures{LetterSpace=10}
        \text_uppercase:n { #1 }
        \group_end:
    }

\NewDocumentCommand{ \UppercaseBold }{ m }
    {
        { \titlefont\UpperCase{#1} }
    }

\NewDocumentCommand{ \PrintUsbLogo }{ m }
    {
        \th_print_logo_head:n { #1 }
    }

\NewDocumentCommand{ \ToC }{}
    {
        \chapter*{Índice~General}
        \@starttoc{toc}
    }

% Notes
\reversemarginpar
\NewDocumentCommand{ \note }{ m }
{
    \marginpar
    {
        \color{blue}
        \raggedleft
        \sffamily
        \scriptsize
        #1
        \PackageWarning{Notes}{Revisar~nota}
    }
}

% Wrapping
\newlength{\wrapwd}
\NewDocumentCommand { \wrapto } { o o m }
{
    \IfNoValueTF { #1 }
        { \setlength{\wrapwd}{.6\linewidth} }
        { \setlength{\wrapwd}{#1} }
    \IfNoValueTF { #2 }
        { \let\wrapal\raggedright }
        { \let\wrapal#2 }
    \begin{minipage}{\wrapwd}
        \wrapal
        #3
    \end{minipage}
}

% Conditionals
\newif { \ifcaratula }
\newif { \ifpaginatitulo }
\newif { \ifresumen }
\newif { \ifdedicatoria }
\newif { \ifagradecimientos }
\newif { \iftoc }
\newif { \ifsimbolos }
\newif { \ifabreviaturas }
\newif { \ifintro }
\newif { \ifbasicos }
\newif { \ifreferencias }

% macros
\newcommand { \MainTitle }
    {
        Estudio~comparativo~de~tres~demostraciones~
        del~teorema~de~inconsistencia~de~Kunen.
    }

\newcommand { \autor }
    {
        Jhonny~Lanzuisi~Berrizbeitia
    }

\newcommand { \tutor }
    {
        Jesús~Nieto~Martínez
    }

\newcommand { \coord }
    {
        Matemáticas
    }

\DeclareMathOperator{\Con}{Con}
\DeclareMathOperator{\Crit}{crit}

\NewDocumentCommand { \model } { m }
{
    \mathfrak { #1 }
}
\NewDocumentCommand { \set } { m }
{
    \left\{ #1 \right\}
}
\NewDocumentCommand { \lex } { m }
{
    \mathcal { #1 }
}
\NewDocumentCommand { \crit } { m }
{
    \Crit (\, #1\, )
}

\NewDocumentCommand { \pwset } { m }
{
    \mathcal{P}(#1)
}

\NewDocumentCommand{ \concept } { m }
{
    \emph{ #1 }
}

\NewDocumentCommand{ \op } { m }
{
     \langle #1 \rangle
}

\NewDocumentCommand{ \con } { m }
{
    \Con ( #1 )
}

\makeatother
\ExplSyntaxOff


\author{\shortname}
\title[\shorttitle]{\MainTitle}

\usetheme[]{Goettingen}
\usecolortheme{dove}

\linespread{1.1}
\setlength{\parskip}{1\baselineskip}

\begin{document}

\begin{frame}
    \hbox
    {
        \begin{minipage}{1.5cm}
            \includegraphics[width=1.5cm]{resources/usblogo}
        \end{minipage}
        \begin{minipage}{\the\dimexpr\linewidth - 1.5cm\relax}
            \footnotesize
            Universidad Simon Bolivar\\
            Coordinacion de Matematicas
        \end{minipage}
    }

    \vspace{\fill}

    \begin{minipage}{\the\dimexpr\linewidth * 8/10\relax}
        \raggedright\large
        \textsc{\MainTitle}
    \end{minipage}

    \vspace{\fill}

    \begin{minipage}{\linewidth}
        \footnotesize
        \begin{minipage}{2.7cm}
            Realizado por:\\
            \shortname
        \end{minipage}
        \begin{minipage}{3cm}
            Con la asesoría de:\\
            \tutor
        \end{minipage}
    \end{minipage}
\end{frame}

\section{Introducción}

\section{Kunen}

\begin{frame}
    \frametitle{Demostración de Kunen}
    \framesubtitle{Funciones $\omega$-Jónsson}

    \begin{defi}
        Sea $x$ un conjunto de ordinales,
        $f\colon [x]^\omega\to x$ es $\omega$-Jónsson
        si, y solo si, para cualquier $y\subseteq x$ tal que $|y|=|x|$
        se tiene $f"[y]^\omega = x$.
    \end{defi}
\end{frame}

\begin{frame}
    \frametitle{Demostración de Kunen}
    \framesubtitle{Funciones $\omega$-Jónsson}

    \begin{teo}
        Para todo $\lambda$, existe una función
        $\omega$-Jónsson para $\lambda$.
    \end{teo}
\end{frame}

\begin{frame}
    \frametitle{Demostración de Kunen}

    Se definen:
    \begin{itemize}
        \item $\kappa=\crit{j}$,
        \item $\lambda=\sup(\set{j^n(\kappa)\mid n\in\omega})$, donde
            $j^0(x)=x$ y $j^{n+1}(x) = j(j^{n}(x))$.
    \end{itemize}

    \pause
    Entonces:
    \begin{itemize}
        \item $j(\lambda) = \lambda$, pues
            $j(\set{j^n(\kappa)\mid n<\omega})
                = \set{j^n(\kappa)\mid 1\leq n<\omega}$
        \item $j^n(\kappa)$ inaccesible $\forall n\in\omega$
        \item $2^\lambda=\lambda^{\aleph_0}$
    \end{itemize}
\end{frame}

\begin{frame}
    \frametitle{Demostración de Kunen}

    En busca de una contradicción, sea $j"\lambda\in M$ y
    $f$ una función $\omega$-Jónsson para $\lambda$.

    \pause
    Entonces:
    \begin{itemize}
        \item $j(f)$ es $\omega$-Jónsson para $j(\lambda)=\lambda$ y,
        \item $j"\lambda\in[\lambda]^\lambda\cap M$.
    \end{itemize}

    \pause
    Si $\lambda = j(f)[j"\lambda]^\omega \subseteq j"\lambda$
    se tendría una contradicción pues $\lambda\neq j"\lambda$.
\end{frame}

\begin{frame}
    \frametitle{Demostración de Kunen}

    Sea $s\in [j"\lambda]^\omega$.

    \pause
    Entonces:
    \begin{itemize}
        \item Existe $t\in [\lambda]^\omega$ tal que $j(t) = s$ y,
        \item $j(f)(s) = j(f)j(t) = j(f(t)) \in j"\lambda$.
    \end{itemize}

    \pause
    Y lo anterior implica $j(f)[j"\lambda]^\omega \subseteq j"\lambda$.
\end{frame}

\section{Woodin}

\begin{frame}
    Math test: $x+y = 22$ some text $\bigcup X_{\alpha}$
\end{frame}

\section{Harada}

\begin{frame}
    Math test: $x+y = 22$ some text $\bigcup X_{\alpha}$
\end{frame}

\section{Conclusiones}

\end{document}
