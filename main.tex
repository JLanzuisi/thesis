% Incluye el preámbulo.
\RequirePackage { geometry }
\RequirePackage { setspace }
\RequirePackage { microtype }
\RequirePackage { enumitem }
\RequirePackage [ spanish ] { babel }
\RequirePackage { csquotes }
\RequirePackage { biblatex }
\RequirePackage { xcolor }
\RequirePackage { hyperref }
\RequirePackage { graphicx }
\RequirePackage { mathtools, amsthm, amsfonts }
\RequirePackage { dirtytalk }
\RequirePackage [ spanish ] { cleveref }
\RequirePackage { epigraph }

\ExplSyntaxOn
\makeatletter

% Cref
\crefname{section}{\S}{\S}

% Set page margins
\geometry
    {
        includehead,
        paper = letterpaper,
        left = 3cm,
        right = 2cm,
        bottom = 2cm,
        top = 2cm,
        marginparsep = 10pt,
        marginparwidth = 2cm,
    }

% Set fonts
\usepackage[lf]{Baskervaldx} % lining figures
\usepackage[bigdelims,vvarbb,baskervaldx]{newtxmath} % math italic letters from Nimbus Roman
\usepackage[cal=boondoxo]{mathalfa} % mathcal from STIX, unslanted a bit
\usepackage[scale=.9]{courierten}
\newcommand{\titlefont}{\bfseries}

% Set basic spacing
\setlength { \parindent } { 8.46pt }
\setlength { \parskip } { 6pt }
\onehalfspacing

% Headers and footer
\pagestyle{headings}
\gdef\@oddhead{\hfil\thepage}

% Bibliography
\addbibresource{references.bib}

% Hyperref
\hypersetup
    {
        colorlinks = true,
        allcolors = black,
    }

% Theorems
\newtheoremstyle{defi}% hnamei
    {.5\baselineskip}% hSpace abovei
    {.5\baselineskip}% hSpace belowi
    {}% hBody fonti
    {}% hIndent amounti
    {\scshape}%\addfontfeatures{LetterSpace=5}}% hTheorem head fonti
    {.}% hPunctuation after theorem headi
    {.5em}% hSpace after theorem headi
    {}% hTheorem head spec (can be left empty, meaning ‘normal’)i
\theoremstyle{defi}
\newtheorem{defi}{Definición}[chapter]

% Sectioning commands
\usepackage[explicit]{titlesec}
\titleformat
    {\chapter}
    [display]
    {\titlefont}
    {
        \centering
        \fontsize{14}{14}\selectfont
        \UpperCase{Capítulo}~\thechapter\\[10pt]
        \fontsize{12}{12}\selectfont
        \UpperCase{#1}
    }
    {0ex}
    {\centering}
\titleformat{name=\chapter, numberless}
    [block]
    {\titlefont}
    {\fontsize{14}{14}\selectfont\UpperCase{#1}}
    {0ex}
    {\centering}
\titleformat
    {\section}
    {\titlefont}
    {\thesection\fontsize{12}{12}\selectfont\ #1}
    {0ex}
    {}
\titleformat{name=\section, numberless}
    [block]
    {\titlefont}
    {\fontsize{12}{12}\selectfont #1}
    {0ex}
    {}

% USB logo
\cs_set_eq:NN \latex_centering:D \centering

\cs_new:Npn \th_print_logo_head:n #1
    {
        \group_begin:
            \latex_centering:D
            \includegraphics[scale=.3]{resources/usblogo} \\
            { \titlefont \UpperCase{#1} }
            \tex_par:D
        \group_end:
    }

\NewDocumentCommand{ \UpperCase }{ m }
    {
        \group_begin:
        %\addfontfeatures{LetterSpace=10}
        \text_uppercase:n { #1 }
        \group_end:
    }

\NewDocumentCommand{ \UppercaseBold }{ m }
    {
        { \titlefont\UpperCase{#1} }
    }

\NewDocumentCommand{ \PrintUsbLogo }{ m }
    {
        \th_print_logo_head:n { #1 }
    }

\NewDocumentCommand{ \ToC }{}
    {
        \chapter*{Índice~General}
        \@starttoc{toc}
    }

% Notes
\reversemarginpar
\NewDocumentCommand{ \note }{ m }
{
    \marginpar
    {
        \color{blue}
        \raggedleft
        \sffamily
        \scriptsize
        #1
        \PackageWarning{Notes}{Revisar~nota}
    }
}

% Wrapping
\newlength{\wrapwd}
\NewDocumentCommand { \wrapto } { o o m }
{
    \IfNoValueTF { #1 }
        { \setlength{\wrapwd}{.6\linewidth} }
        { \setlength{\wrapwd}{#1} }
    \IfNoValueTF { #2 }
        { \let\wrapal\raggedright }
        { \let\wrapal#2 }
    \begin{minipage}{\wrapwd}
        \wrapal
        #3
    \end{minipage}
}

% Conditionals
\newif { \ifcaratula }
\newif { \ifpaginatitulo }
\newif { \ifresumen }
\newif { \ifdedicatoria }
\newif { \ifagradecimientos }
\newif { \iftoc }
\newif { \ifsimbolos }
\newif { \ifabreviaturas }
\newif { \ifintro }
\newif { \ifbasicos }
\newif { \ifreferencias }

% macros
\newcommand { \MainTitle }
    {
        Estudio~comparativo~de~tres~demostraciones~
        del~teorema~de~inconsistencia~de~Kunen.
    }

\newcommand { \autor }
    {
        Jhonny~Lanzuisi~Berrizbeitia
    }

\newcommand { \tutor }
    {
        Jesús~Nieto~Martínez
    }

\newcommand { \coord }
    {
        Matemáticas
    }

\DeclareMathOperator{\Con}{Con}
\DeclareMathOperator{\Crit}{crit}

\NewDocumentCommand { \model } { m }
{
    \mathfrak { #1 }
}
\NewDocumentCommand { \set } { m }
{
    \left\{ #1 \right\}
}
\NewDocumentCommand { \lex } { m }
{
    \mathcal { #1 }
}
\NewDocumentCommand { \crit } { m }
{
    \Crit (\, #1\, )
}

\NewDocumentCommand { \pwset } { m }
{
    \mathcal{P}(#1)
}

\NewDocumentCommand{ \concept } { m }
{
    \emph{ #1 }
}

\NewDocumentCommand{ \op } { m }
{
     \langle #1 \rangle
}

\NewDocumentCommand{ \con } { m }
{
    \Con ( #1 )
}

\makeatother
\ExplSyntaxOff


% Secciones a incluir.
\caratulafalse
\paginatitulotrue
\resumentrue
\dedicatoriafalse
\agradecimientosfalse
\listastrue
\toctrue
\introtrue
\referenciastrue
\basicostrue
\mainproofstrue

% TODO:
% demostrar 0.1 del kanamori para citarlo en la demostración del teorema de Kunen
% de Woodin.

\begin{document}
% ------------------------------------------------------------------------------
%                                 FRONTMATTER.
% ------------------------------------------------------------------------------
\frontmatter
% ------------------------------------------------------------------------------
% Caratula.
% ------------------------------------------------------------------------------
\ifcaratula\newpage
WIP: CARATULA.
\fi
% ------------------------------------------------------------------------------
% Página de Título.
% ------------------------------------------------------------------------------
\ifpaginatitulo\newpage
\bgroup
    \centering
    \thispagestyle{empty}

    \PrintUsbLogo
    {
        Universidad Simón Bolívar\\
        Decanato de estudios profesionales\\
        Coordinación de \coord
    }

    \vspace{1.5cm}

    \UppercaseBold{
        \wrapto[14cm][\centering]
        {\MainTitle}
    }

    \vspace{1.5cm}

    Por:
    \\
    \autor

    \vspace{1.5cm}

    Realizado con la asesoría de:
    \\
    \tutor

    \vspace{3cm}

    \MakeUppercase{Proyecto de Grado}\\
    Presentado ante la Ilustre Universidad Simón Bolívar\\
    como requisito parcial para optar al título de\\
    Licenciatura en Matemáticas Puras

    \vspace{\fill}

    \textbf{Sartenejas, \today}\par
\egroup
\fi
% ------------------------------------------------------------------------------
% Resumen.
% ------------------------------------------------------------------------------
\ifresumen\newpage
\PrintUsbLogo
    {
        Universidad Simón Bolívar\\
        Decanato de estudios profesionales\\
        Coordinación de \coord
    }

\begin{center}
    \begin{minipage}{14cm}
        \centering
        \UppercaseBold
            {
                \MainTitle
            }
    \end{minipage}

    \vspace{.5cm}

    \UpperCase
        {
            Proyecto de grado
        } \\
    Realizado por: \autor \\
    Con la asesoría de: \tutor \\[.5cm]
    \UppercaseBold{Resumen}

\end{center}

El teorema de inconsistencia de Kunen, que establece la inexistencia en
ZFC (teoría de conjuntos Zermelo-Fraenkel con el axioma de elección)
de cualquier inmersión elemental $j\colon V\embed V$ y, por lo tanto, de los cardinales
de Reinhardt, es un resultado central de la teoría de cardinales grandes
debido a que establece una cota superior para dicha teoría.

Se tratará el teorema mencionado a través de tres demostraciones, cada una de naturaleza
distinta: primero aquella dada por Kunen originalmente, relacionada a la combinatoria infinita,
luego otra debida a Hugh Woodin concerniente a conjuntos estacionarios y finalmente una de
Mikio Harada. El libro de Akihiro Kanamori \autocite{cohen_independence_1964} es la referencia
estándar en el estudio de los cardinales grandes y la fuente de dichas demostraciones.

Para poder estudiar el resultado de Kunen en profundidad,
se divide el presente escrito en 3 capítulos más la introducción.
En la introducción se discuten los antecedentes históricos y la importancia
de este teorema.
El primer capítulo consta de nociones básicas necesarias para su enunciación y demostración.
El segundo capítulo se encarga de enunciar el teorema de Kunen y dar sus demostraciones.
Finalmente, en el tercer capítulo, se discuten resultados recientes relacionados al teorema de Kunen
y su problema abierto asociado:
¿Seguirá siendo cierto el resultado de Kunen si se prescinde del axioma de elección?

\vspace{\fill}

\textbf{Palabras Clave:}
\fi
% ------------------------------------------------------------------------------
% Dedicatoria.
% ------------------------------------------------------------------------------
\ifdedicatoria\newpage
WIP: DEDICATORIA.
\fi
% ------------------------------------------------------------------------------
% Agradecimientos.
% ------------------------------------------------------------------------------
\ifagradecimientos\newpage
WIP: AGRADECIMIENTOS.
\fi
% ------------------------------------------------------------------------------
% Listas de símbolos y abreviaturas.
% ------------------------------------------------------------------------------
\iflistas
\ChapterNoNumber{Lista de Símbolos}

En la lista siguiente, $C$ es un conjunto y $\gamma$ un ordinal.

\begin{center}
    \begin{tabular}{cl}
        Símbolo & Significado \\
        \hline\noalign{\smallskip}
        $\pwset{C}$ & Conjunto de partes.\\
        $\sup(\gamma)$ & Supremo, es decir, $\bigcup \gamma$.\\
        $\cf(\gamma)$ & Cofinalidad de $\gamma$.\\
        $\ot(\gamma)$ & Tipo de orden de $\gamma$.
    \end{tabular}
\end{center}

\newpage
\ChapterNoNumber{Lista de Abreviaturas}

\begin{center}
    \begin{tabular}{cl}
        Abreviatura & Significado \\
        \hline\noalign{\smallskip}
        ZF & Teoría de conjuntos de Zermelo-Fraenkel.\\
        AC & Axioma de elección.\\
        ZFC & ZF al añadir AC.\\
        NBG & Teoría de conjuntos de Von Neumann, Bernays y Gödel.\\
        CH & Hipótesis del continuo: $2^{\aleph_0} = \aleph_1$. \\
        \cna{} & Cerrado no acotado. \\
    \end{tabular}
\end{center}
\fi
% ------------------------------------------------------------------------------
% Tabla de contenidos.
% ------------------------------------------------------------------------------
\iftoc \ToC \fi
% ------------------------------------------------------------------------------
%                                 MAINMATTER.
% ------------------------------------------------------------------------------
\mainmatter
% ------------------------------------------------------------------------------
% Introducción.
% ------------------------------------------------------------------------------
%     - Conceptos básicos de teoría de conjuntos. Establecer la teoría de conjuntos en la que se va a trabajar y establecer notación.
% Citar CG de ivorra en el parrafo sobre la importancia.
% Mencionar la jerarquía de los CG, y explicar que se tratará una parte superior
% de la jerarquía: de los cardinales medibles para arriba. Quizás dar un breve resúmen
% de como es la jerarquía por debajo. (Los grafos que tiene Ivorra en CG podrían usarse)
\ifintro
\ChapterNoNumber{Introducción}
\thispagestyle{empty}

\epigraph
{
Del paraíso que Cantor ha creado para nosotros, nadie ha de expulsarnos.
}
{David Hilbert. \autocite[170]{hilbert_uber_1926}}

Las hipótesis de cardinales grandes son los axiomas matemáticos
más fuertes jamás postulados: estipulan la existencia de conjuntos
infinitos de tal tamaño, que no son decidibles en el marco de la teoría de conjuntos.
Los más pequeños entre ellos, siendo enormes, nunca son suficientemente
fuertes para demostrar la existencia de cardinales mayores.
El teorema de inconsistencia de Kunen es
una cota superior que impone el axioma de elección a las
hipótesis de cardinales grandes.
Vale la pena preguntarse: ¿Qué utilidad pueden tener los cardinales grandes?,
¿por qué interesa el resultado de Kunen?

La principal razón por la que el estudio de estos conjuntos infinitos
es relevante proviene del siguiente hecho:
es con ellos que cualquier aserción sobre consistencia relativa
puede medirse.
Sabemos gracias a Gödel que si en una teoría se puede desarrollar la aritmética
elemental, esta no puede demostrar su propia consistencia.
Si $T$ es una teoría de esta forma, lo que sí se puede es construir la teoría
$T'=T+\con{T},$
donde añadimos como nuevo axioma la consistencia
de $T$,
que evidentemente demuestra que $T$ es consistente; todo esto sin contradecir
el resultado de Gödel. Pero tenemos ahora un nuevo problema: la consistencia
de $T'$.
Consideremos entonces otra teoría,
\[T'' = T + \con{T} + \con{T + \con{T}},\]
que demuestra la consistencia de $T'$.
Podemos continuar de esta manera, definiendo $T''', T'''',\text{etc}$.
De esta manera se construye una jerarquía análoga a la de los ordinales,
en una torre ascendente infinita de consistencia relativa
\autocite[\S 7.7]{hamkins_lectures_2020}.

La conexión importante es la siguiente:
los cardinales grandes representan una instanciación de esta jerarquía.
Podemos entonces, a través de ellos, estudiar este universo infinito de
consistencia al que Gödel nos abrió las puertas.

\section*{Cardinales Grandes. Extensión hasta la Inconsistencia.}

Los cardinales grandes tienen sus orígenes en las investigaciones cantorianas
sobre conjuntos definibles de números reales y los números transfinitos.
Fue Felix Hausdorff \autocite{hausdorff_grundzuge_nodate}
el primero en considerar un cardinal grande,
los débilmente inaccesibles.
Paul Mahlo \autocite{mahlo_uber_1911,mahlo_zur_1912,mahlo_zur_1913}
postulará después los cardinales que llevan su nombre.
Al considerar la clausura sobre la formación del conjunto de partes,
Sierpiński-Tarski \autocite{sierpinski_sur_1930} y  Zermelo \autocite{zermelo_uber_1930}
llegan a la noción de cardinal (fuertemente) inaccesible.

Stanisław Ulam \autocite{ulam_zur_1930}, al estudiar la medida de Lebesgue,
introduce los cardinales medibles y con ellos la primera
pregunta sobre la jerarquía de los cardinales grandes:
¿Es el primer cardinal inaccesible también medible?
El desarrollo de los cardinales grandes dependerá a partir de este
momento de la incorporación de la teoría de modelos (\ref{sec:models}) en las matemáticas.

La generalización de la lógica de primer orden, obtenida al
permitir una cantidad infinita de operaciones lógicas,
permitió a Tarski \autocite{tarski_problems_1966} definir los cardinales (débil y fuerte)
compactos como una generalización del teorema de compacidad
para estas lógicas. Los cardinales compactos dieron solución
a la pregunta propuesta unos párrafos más arriba:
el primer cardinal inaccesible no es medible.

El siguiente gran salto adelante vendría de la mano de
Paul Cohen \autocite{cohen_independence_1963,cohen_independence_1964}
y la invención del forcing como técnica
para establecer resultados de consistencia relativa.
Cohen usaría su nueva técnica para construir un modelo de la
teoría de conjuntos donde falla la hipótesis del continuo
y junto con un resultado anterior de Gödel---a saber, que
en el universo de los constructibles se verifica CH---%
logra resolver finalmente la gran pregunta de Cantor sobre cardinalidades
intermedias entre los naturales y el continuo.

Finalmente, en la década de 1970,
Solovay y Reinhardt comienzan a postular
hipótesis de cardinales grandes aún más fuertes
que las anteriores.
Al poner en el centro el concepto de inmersión elemental
(\ref{sec:elem-embed}),
nacen las nociones de cardinal supercompácto y extendible.

Reinhardt \autocite{reinhardt_ackermanns_1970},
generalizando su concepto de extendibilidad,
propone el mayor principio de reflección posible:
la existencia de una inmersión elemental $j\colon V\embed V$
y la consideración de $\crit{j}$ como cardinal grande.

Es aquí que irrumpe el resultado de Kunen,
estableciendo la imposibilidad de dicha inmersión
y delimitando por arriba la jerarquía de los cardinales
grandes. A partir de este momento, el desarrollo
de esta teoría se dará considerando cardinales más
débiles que el propuesto por Reinhardt,
para evitar la inconsistencia.

Al momento de demostrar este resultado, Kunen hace
uso del axioma de elección. Como se verá
más adelante, todas las demostraciones dadas en este texto
dependerán del axioma de elección.
Es natural entonces preguntarse: ¿Realmente se necesita AC?,
¿Es demostrable el teorema de Kunen en ZF?

Esta última pregunta es, actualmente, un problema abierto.
Lo que indica una posible vía por la que se puede desarrollar
el estudio del resultado de Kunen, y muestra de que más allá
de la potencia de dicho teorema, quedan aún preguntas por explorar.

\fi
% ------------------------------------------------------------------------------
% Cap 1: Preliminares.
% ------------------------------------------------------------------------------
\ifbasicos
% - Nociones de filtro, ultrafiltro, ideal y filtros k-completos.
% - Conjuntos cerrados no acotados y estacionarios.
% - Teoría de modelos, conceptos básicos.
% - Inmersiones elementales y puntos críticos, haciendo mención de cardinales medibles.
\chapter{Nociones básicas}
\thispagestyle{empty}

Este capítulo establece varios conceptos básicos que serán necesarios
más adelante. Las nociones de filtro, ultrafiltro y filtro $\kappa$-completo
junto con los conjuntos no acotados y estacionarios componen las definiciones de
conjuntos más elementales que harán falta.
Luego, un rápido repaso de la teoría de modelos permitirá abordar las inmersiones
elementales, que son una pieza central del teorema de Kunen.

Es bien sabido que existen diversos sistemas axiomáticos
con los cuales se puede desarrollar la teoría de conjuntos.
En todo este texto, se usará el de Zermelo-Fraenkel con el axioma de elección, tal como
aparece en cualquiera de las referencias estándar \autocite{kunen_set_2013,jech_set_2003}.
Más aún, se asume familiaridad con las nociones elementales de la teoría de conjuntos
y de la lógica de primer orden.

\section{Filtros}

Esta sección se ocupa de dar las definiciones básicas de filtros,
que serán necesarias a lo largo del texto.
Los filtros caracterizan a conjuntos ``grandes'' dentro
de un conjunto dado $C$.

\begin{defi}
    Sea $C$ un conjunto no vacío. Un conjunto $F\subset \pwset{C}$ es un
    \concept{filtro} si se cumplen las siguientes condiciones:
    \begin{enumerate}[label=(\roman*)]
        \item $C\in F$ y $\emptyset\notin F$.
        \item Si $X,Y\in F$ entonces $X\cap Y\in F$.
        \item Si $X,Y\subset C$, $X\in F$ y $X\subset Y$ entonces $Y\in F$.
    \end{enumerate}
\end{defi}

\begin{defi}
    Sea $F$ un filtro sobre $C$. $F$ es \concept{ultrafiltro} si, para todo $X\subset C$,
    se tiene que $X\in F$ o $C-X\in F$.
\end{defi}

Una caracterización para ultrafiltros viene dada por la propiedad
de maximalidad:

\begin{teo}
    Sea $F$ un filtro sobre $C$. $F$ es \concept{ultrafiltro} si, y solo si, es maximal.
\end{teo}

La siguiente definición es central para la teoría de cardinales medibles.

\begin{defi}
    Sea $\kappa$ un cardinal regular y $F$ un filtro sobre $C$.
    $F$ es $\kappa$-completo siempre que dada una familia de conjuntos
    $\set{X_\alpha\in F\mid \alpha<\kappa}$,
    se tiene que
    \[
        \bigcap X_\alpha \in F.
    \]
\end{defi}

Un ejemplo que une los conceptos tratados hasta ahora es, como ya se mencionó,
la definición de cardinal medible.

\begin{defi}
    Sea $\kappa > \omega$ un cardinal. $\kappa$ es \concept{medible} si existe
    un ultrafiltro $\kappa\text{-completo}$ sobre $\kappa$.
\end{defi}

\section{Conjuntos Estacionarios}

El principal objetivo de esta sección es establecer un teorema
de Solovay, acerca de particiones
con conjuntos estacionarios, usando el teorema \ref{teo:fodor}
de Fodor.

Sea $C$ un conjunto y $X\subset C$, diremos que $X$ es \concept{no acotado}
en $C$ si $\sup(X) = C$.
Si $C$ es además un conjunto de ordinales, un ordinal límite $\alpha$ es
\concept{punto límite} de $C$ si $\sup ( C \cap\alpha ) = \alpha$.
\begin{defi}
    Sea $\kappa$ un cardinal regular no numerable. Un conjunto $C\subset \kappa$
    es \concept{cerrado no acotado} (\cna) si $C$ es no acotado en $\kappa$ y contiene a
    todos sus puntos límites menores que $\kappa$. En particular, un conjunto es $\nu$-cerrado
    para $\nu<\kappa$ si contiene a todos sus puntos límite menores que $\kappa$ de cofinalidad $\nu$.
    Un conjunto $S\subset\kappa$ es \concept{estacionario} si para cada conjunto
    \cna{} $C\subset\kappa$ se tiene $S\cap C\neq\emptyset$.
\end{defi}

Será de utilidad saber el comportamiento de los conjuntos \cna{} bajo intersecciones.
Para este fin, definimos, dada $\op{X_\alpha\mid \alpha<\kappa}$ una sucesión
de subconjuntos de $\kappa$, la \concept{intersección diagonal} de
$X_\alpha$ como:
\[
    \dint_{\alpha<\kappa}X_\alpha
    =
    \set{\epsilon<\kappa\mid\epsilon\in \bigcap_{\alpha<\epsilon}X_\alpha} .
\]

\begin{teo}\label{teo:intersection-cna}
    Sea $\kappa$ un cardinal regular no numerable y $\set{C_\alpha}_{\alpha<\kappa}$ una familia
    de \cna{} en $\kappa$, entonces:
    \begin{enumerate}[label=(\roman*)]
        \item $C_\alpha\cap C_\beta$ es \cna{} ($\alpha,\beta < \kappa$).
        \item $\bigcap_{\alpha<\kappa}C_\alpha$ es \cna.
        \item $\dint_{\alpha<\kappa}C_\alpha$ es \cna.
    \end{enumerate}
\end{teo}

\begin{proof}\phantom{a}
    \begin{enumerate}[label=(\roman*)]
        \item\label{pr:intersection-simple-cna}
            Es claro que $C\cap D$ es cerrado. Veamos que es no acotado.
            Sea $\alpha<\kappa$. Dado que $C$ es no acotado, existe $\alpha_1\in C$
            tal que $\alpha_1 > \alpha$. De la misma forma, existe $\alpha_2\in D$
            tal que $\alpha_2 > \alpha_1$. Podemos seguir con este proceso para obtener
            una sucesión creciente:
            \[
                \alpha < \alpha_1 < \alpha_2 < \dots
            \]
            Sea $\beta$ el límite de la sucesión de arriba.
            Entonces $\beta < \kappa$ y $\beta\in C$ y $\beta\in D$.


        \item\label{pr:intersection-cna}
            La demostración será por inducción.
            Sea $\lambda<\kappa$ y $\seq{C_\alpha\mid\alpha<\lambda}$
            una sucesión de conjuntos \cna{} en $\kappa$.
            Para los ordinales sucesores, podemos simplemente aplicar
            \ref{pr:intersection-simple-cna}.
            Si $\lambda$ es ordinal límite, asumiremos que el teorema
            es cierto para cada $\alpha<\lambda$. Podemos ahora sustituir
            cada $C_\alpha$ por $\bigcap_{\xi\leq\alpha} C_\xi$ y obtenemos
            una sucesión decreciente con la misma intersección. Entonces a partir de ahora:
            \[
                C_0 \subset C_1 \subset C_2 \subset \dots
            \]
            serán \cna{} y $C = \bigcap_{\alpha<\lambda} C_\alpha$.
            Por la misma razón que \ref{pr:intersection-simple-cna}, no es difícil
            ver que $C$ es cerrado. Veamos que es no acotado. Sea $\alpha<\kappa$,
            construiremos una sucesión de la siguiente forma: sea $\beta_0\in C_0$ mayor que
            $\alpha$, y para cada $\xi<\lambda$ se tomará $\beta_\xi\in C_\xi$
            tal que $\beta_\xi > \sup\set{\beta_\nu\mid\nu<\xi}$.
            Dado que $\kappa$ es regular y $\lambda<\kappa$, la sucesión que se acaba de
            describir existe y su límite $\beta$ es menor que $\kappa$.
            Para cada $\eta<\lambda$, $\beta$ es límite de una sucesión
            $\seq{\beta_\xi\mid\eta\leq\xi<\lambda}$ en $C_\eta$, por lo que
            $\beta\in C_\eta$ y esto implica $\beta\in C$.


        \item Llamemos $D$ a $\dint_{\alpha<\kappa}C_\alpha$. Veamos primero que $D$ es cerrado.
            Sea entonces $\lambda<\kappa$ tal que $D\cap\lambda$
            no está acotado en $\lambda$, esto es, que $\lambda$ es punto límite de $D$.
            Tomemos $\beta\in\lambda$,
            entonces existe $\epsilon\in\lambda\cap D$ tal que $\beta<\epsilon$ pues
            $D\cap\lambda$ es no acotado.
            Como $\epsilon\in D$, existe $C_\alpha$, con $\alpha<\epsilon<\lambda$,
            al que $\epsilon$ pertenece.
            Pero entonces, lo que hemos demostrado es que siempre que tomemos $\beta\in\lambda$
            existe $\epsilon\in C_\alpha\cap\lambda$ que esta por encima de $\beta$ o, equivalentemente,
            que $C_\alpha\cap\lambda$ es no acotado en $\lambda$.
            Al ser $C_\alpha$ cerrado tenemos $\lambda\in C_\alpha$ y esto implica
            $\lambda\in D$. Luego $D$ es cerrado.

            Solo falta ver que $D$ es no acotado en $\kappa$.
            Para esto notemos que, debido a \ref{pr:intersection-cna},
            se puede reemplazar cada $C_\alpha$ por $\bigcap_{\xi\leq\alpha} C_\xi$
            y obtenemos una sucesión decreciente $C_0\subset C_1\subset\dots$
            que no cambia el valor de $D$.
            Sea $\gamma\in\kappa$. Como cada $C_\alpha$ es no acotado en $\kappa$,
            podemos construir una sucesión $\seq{\beta_n\mid n\in\omega}$ de la siguiente forma:
            tomamos $\beta_0\in C_0$ mayor que $\gamma$, luego dado $\beta_n$, tomamos
            $\beta_{n+1} \in C_{\beta_n}$ mayor que $\beta_n$. Llamemos $\beta = \lim_n\beta_n$
            y tomemos $\xi<\beta$. Entonces existe $\beta_n>\xi$ y cada $\beta_k$ con $k>n$
            pertenece a $C_{\beta_n}$, pues los $C_\alpha$ están encajados,
            por lo que $\beta\in C_{\beta_n}$ y $\beta\in C_\xi$.
            Pero esto muestra que $\beta\in D$ y que $D$ es no acotado.
    \end{enumerate}
\end{proof}

\begin{teo}\label{teo:stationary-intersect}
    Sea $\lambda>\omega$ regular y $\nu<\lambda$ también regular.
    Si $S\subseteq\set{\xi<\lambda\mid \cf(\xi)=\omega}$ es estacionario en $\lambda$
    y $C$ es un conjunto $\nu$-cerrado no acotado en $\lambda$, entonces $S\cap C\neq\emptyset$.
\end{teo}

\begin{proof}
\end{proof}

\begin{defi}
    Una función de ordinales $f$ en un conjunto $S$ es \concept{regresiva}, si
    $f(\alpha)<\alpha$ para todo $\alpha\in S$.
\end{defi}

\begin{teo}[Fodor]\label{teo:fodor}
    Sea $f$ una función regresiva en un conjunto estacionario $E\subset\kappa$.
    Entonces existe $\alpha\in\kappa$ tal que $f^{-1}(\set{\alpha})$ es estacionario.
\end{teo}

\begin{proof}
    Supongamos, en busca de una contradicción, que $f^{-1}(\set{\alpha})$ no es estacionario
    para todo $\alpha<\kappa$. Entonces existen conjuntos \cna{} $C_\alpha$ tales que
    $C_\alpha\cap f^{-1}(\set{\alpha}) = \emptyset$, esto es,
    que $f(\gamma)\neq\alpha$ para todo $\gamma\in E\cap C_\alpha$.
    Si $D=\dint_{\alpha<\kappa} C_\alpha$,
    por \ref{teo:intersection-cna}, $D$ es \cna{} en $\kappa$.
    Pero entonces $D\cap E\neq\emptyset$ y podemos tomar $\gamma\in D\cap E$,
    luego, $f(\gamma)\neq\alpha$ para todo $\alpha<\gamma$
    lo que implica $f(\gamma)\geq\gamma$ y esto es una contradicción.
\end{proof}

El siguiente es un teorema auxiliar, que será de utilidad para \ref{teo:solovay}.

\begin{teo}\label{teo:stationary}
    Sea $E\subset\kappa$ un conjunto estacionario en $\kappa$ y supongamos que todo
    ordinal perteneciente a $E$ es regular no numerable. Entonces el conjunto
    \[
        T=\set{\alpha\in E\mid
            E\cap\alpha\;\text{no es un subconjunto estacionario de $\alpha$}}
    \]
    es estacionario en $\kappa$.
\end{teo}

\begin{proof}
    Veamos que $T$ intersecta a todos los \cna{} de $\kappa$.
    Sea $C$ \cna{} en $\kappa$ y $C'$ el subconjunto de los puntos límite de $C$.
    Tenemos que $C'$ también es \cna{} en $\kappa$ por lo que podemos tomar
    el menor $\alpha\in C'\cap E$.
    Puesto que $\alpha$ es regular y punto límite de $C$, $C_\alpha\cap\alpha$ es un subconjunto
    \cna{} de $\alpha$, como también lo es $C'\cap\alpha$. Dado que $\alpha$ es el elemento
    más pequeño de $C'\cap E$, $C'\cap E\cap\alpha = \emptyset$. Esto último
    dice que $E\cap\alpha$ es no estacionario en $\alpha$, y $\alpha\in T\cap C$.
\end{proof}

\begin{teo}[Solovay]\label{teo:solovay}
    Sea $\kappa$ un cardinal regular no numerable. Entonces cada subconjunto estacionario
    de $\kappa$ es la unión disjunta de $\kappa$ subconjuntos estacionarios.
\end{teo}

\newcommand{\seqa}{a_\xi^\alpha}
\begin{proof}
    Sea $E$ un subconjunto estacionario de $\kappa$.
    Por \ref{teo:stationary}, asumiremos que el conjunto $W$
    consistente de todos los $\alpha\in E$ tales que $\alpha$ es cardinal
    regular y $E\cap\alpha$ no es estacionario en $\alpha$, es estacionario en $\kappa$.
    Existe entonces un conjunto \cna{} $C_\alpha\subset\alpha$ tal que $E\cap C_\alpha = \emptyset$,
    pero $W\subset E$ por lo que $C_\alpha\cap W=\emptyset$.
    Sea $\seq{\seqa\mid\xi<\alpha}$ una enumeración creciente de $C_\alpha$.
    Se tiene entonces que $\lim_{\xi\to\alpha}\seqa = \alpha$ y $\seqa\notin W$ para todo $\xi, \alpha$.

    Veamos, en primer lugar, que existe $\xi$ tal que, para todo $\eta<\kappa$, el conjunto:
    \begin{equation}\label{eqn:solovay-stationary}
        \set{\alpha\in W\mid \seqa\geq\eta}
    \end{equation}
    es estacionario. Si este no fuese el caso, para cada $\xi$ tendríamos un $\eta(\xi)$
    y un conjunto \cna{} $C_\xi$, tal que $\seqa < \eta(\xi)$ para todo $\alpha\in W\cap C_\xi$, siempre que $\seqa$
    esté definida. Sea $C$ la intersección diagonal de los $C_\xi$. Entonces si $\alpha$ es un elemento de
    $W\cap C$, se tiene que $\seqa < \eta(\xi)$ para todo $\xi<\alpha$. Consideremos ahora el conjunto $D$ de
    los $\gamma\in C$ tales que $\eta(\xi)<\gamma$ para todo $\xi<\gamma$, este conjunto es \cna{}
    y $W\cap D$ es estacionario. Sean $\alpha<\gamma$ dos ordinales en $W\cap D$,
    si $\xi<\gamma$ entonces $\seqa<\eta(\xi)<\gamma$, lo cual implica que $a_\gamma^\alpha = \gamma$.
    Pero esto es una contradicción, puesto que $\gamma\in W$ y $a_\gamma^\alpha\notin W$.

    Tenemos ahora $\xi$ tal que \ref{eqn:solovay-stationary} es estacionario.
    Sea $f$ una función en $W$ definida por $f(\alpha)=\seqa$. Por la definición de $\seqa$
    la función $f$ es regresiva, por lo que para cada $\eta<\kappa$ el teorema \ref{teo:fodor} de Fodor
    nos da un conjunto estacionario $E_\eta$ de \ref{eqn:solovay-stationary} y un $\gamma_\eta\geq\eta$
    que es testigo de que $E_\eta$ sea estacionario.
    Ahora, si $\gamma_\eta\neq\gamma_{\eta'}$ entonces $E_\eta\cap E_{\eta'} = \emptyset$ y, puesto que $\kappa$ es regular,
    se tiene también $|\set{E_\eta\mid\eta<\kappa}| = |\set{\gamma_\eta\mid\eta<\kappa}| = \kappa$.
\end{proof}

\section{Teoría de Modelos}
\label{sec:models}

La teoría de modelos es un área relativamente joven \autocite[3]{chang_model_2012}.
No obstante, su desarrollo ha sido crucial para la teoría de conjuntos y los
cardinales grandes \autocite[xv]{kanamori_higher_2009}.

Se quiere definir lo que es un modelo para un lenguaje formal $\lex{L}$.
Un lenguaje $\lex{L}$ es un conjunto de símbolos relacionales, funcionales y constantes.
Los símbolos relacionales y funcionales pueden tener cualquier cantidad finita de argumentos,
lo que se conoce usualmente como su aridad, excepto cero.

Dado un conjunto cualquiera $A$, interesa darle significado a los símbolos de un
lenguaje $\lex{L}$ en $A$. Esto se logra a través de una \concept{interpretación}, esto es,
una correspondencia que asigna a cada relación $n$-aria $P$ una relación
$R\subset A^n$, a cada función $m$-aria una función $G\colon A^m\to A$ y a cada
constante $c$ un elemento $x\in A$.

\begin{defi}\label{def:model}
    Sea $\lex{L}$ un lenguaje formal. Un \concept{modelo} $\model{A}$ para $\lex{L}$ se define como,
    \[
        \model{A} = \op{A, \mathcal{I}}.
    \]
    Donde $A$, que es un conjunto cualquiera, es el \concept{universo} de $\model{A}$ y
    $\mathcal{I}$ es una interpretación de los símbolos de $\lex{L}$ en $A$.
\end{defi}

Dada una sentencia $\phi$ de un lenguaje $\lex{L}$ y $\model{A}$ un modelo para $\lex{L}$,
se escribirá $\model{A}\models\phi$ si la fórmula $\phi$ se satisface en $\model{A}$.
Intuitivamente, la relación $\models$ quiere decir que $\phi$ es verdadera en el modelo.
Una definición rigurosa de $\models$ es posible, y requiere inducción sobre la complejidad
de $\phi$ (véase \autocite[\S 1.3]{chang_model_2012} o \autocite[\S 12]{jech_set_2003}).

Dados dos modelos $\model{A}, \model{B}$ se dirá que $\model{A}$ es \concept{elementalmente
    equivalente} a $\model{B}$, en símbolos $\model{A}\equiv \model{B}$, si toda sentencia
que es verdadera en $\model{A}$ lo es también en $\model{B}$ y viceversa.

La definición \ref{def:model} está dada de manera general. Normalmente interesarán modelos
del lenguaje de la teoría de conjuntos, denotado $\lst$, el cual consiste de la lógica de
primer orden con la relación de igualdad y el símbolo binario $\in$.
Los $\in$-modelos de la forma $\op{A,\in}$, a los que denotaremos
solamente por $A$, son los modelos de $\lst$ con los que se trabajará la mayoría del tiempo.
Existe una clase de $\in$-modelos de gran importancia, que se definen a continuación.

\begin{defi}
    Un \concept{modelo interno} de ZF es un $\in$-modelo transitivo
    donde se satisfacen los axiomas y que contiene a los ordinales.
\end{defi}

\section{Inmersiones Elementales}
\label{sec:elem-embed}

El objetivo de este capítulo es establecer los resultados básicos
sobre las inmersiones elementales de modelos internos de ZFC.

\begin{defi}
    Sean $\model{M} = \op{M,\dots}$ y $\model{N} = \op{N,\dots}$ dos modelos de un lenguaje $\lex{L}$.
    Una función inyectiva $f\colon M\to N$ es una \concept{inmersión elemental},
    denotado por $f\colon \model{M}\embed\model{N}$, si, y solo si, para cualquier fórmula
    n-aria $\phi$ de $\lex{L}$ y $x_1,\dots,x_n \in M$,
    \[
        \model{M}\models\phi(x_1,\dots,x_n) \iff \model{N}\models\phi(f(x_1),\dots,f(x_n)).
    \]
    Si $f$ es la función identidad, diremos que $\model{M}$
    es una subestructura elemental de $\model{N}$ y se denotará por $\model{M}\embed\model{N}$.
\end{defi}

Hace falta una pequeña digresión para tratar el caso de inmersiones elementales
entre clases propias transitivas. Es sabido que en ZFC no es posible formalizar
el concepto de inmersión elemental para clases propias, pues lo prohíbe el teorema
de la indefinibilidad de la verdad de Tarski.
A partir de ahora la noción de inmersión elemental se trabajará de manera informal,
pero sin olvidar que, en los contextos que será utilizada, puede ser formalizada
en ZFC \autocite[45-46]{kanamori_higher_2009}.

De la definición de inmersión elemental se sigue que estas preservan todas
las operaciones conjuntistas que son absolutas para modelos transitivos.
En particular, las inmersiones envían ordinales en ordinales y
preservan su orden.

\begin{teo}\label{teo:elem-embed-trivial}
    Sean $M$ y $N$ modelos internos de ZFC y $j\colon M\embed N$.
    Si $j$ no es la función identidad, existe un ordinal $\delta$
    tal que $j(\delta)>\delta$.
\end{teo}

\begin{proof}
    Primero, $j(\delta)$ nunca es estrictamente menor que $\delta$:
    si este fuese el caso, podríamos tomar el menor $\delta$ con dicha propiedad
    y puesto que $j(\delta) < \delta \in M$, y $M$ transitivo, se tendría
    $j(\delta)\in M$ y al considerar ahora $j(j(\delta))$ se llega a la conclusión
    $j(j(\delta)) < j(\delta)$, pues las inmersiones preservan el orden.

    Sea $x\in M$ y $b = \tc(\set{x})$ su clausura transitiva en $V$.
    Supongamos que $j(\delta) = \delta$ para todo ordinal $\delta\in M$.
    Si $x\in M$ es un conjunto de ordinales entonces $j(x)=x$.
    Dado que $M\models \text{AC}$, existe un ordinal $\gamma$ y
    una biyección $e\in M$ que va de $\gamma$ sobre $b$.
    Sea $E\in M$ la relación binaria sobre $\gamma$ definida por:
    \[
        \op{\alpha, \beta}\in E\quad\text{si, y solo si,}\quad e(\alpha)\in e(\beta).
    \]
    Se puede identificar a $E$ con un conjunto de ordinales de la forma usual
    para obtener $j(E)=E$. Puesto que todo subconjunto no vacío de $\gamma$
    tiene un elemento $E$-minimal en $V$, se sigue que esto también ocurre en $M$ y $N$
    y que $E$ está bien fundada en ambos conjuntos.
    Se puede entonces usar el teorema de colapso de Mostowski para $\op{\gamma, E}$
    tanto en $M$ como en $N$ para obtener un isomorfismo entre $\op{\gamma, E}$
    y $\op{M,\in}$ donde $M$ es transitivo, pero, como el colapso transitivo
    es único, debe ocurrir $b = M$.

    Se sigue del párrafo anterior que $j(b)=b$, en efecto,
    la elementalidad de $j$ junto con $j(E)=E$ y el hecho de que
    $\op{b,\in}$ es el colapso transitivo único de $\op{\gamma, E}$
    tanto en $M$ como en $N$, obligan a que $j(b)=b$.
    Pero $x$ es definible como el elemento de mayor rango de $b$, por lo que también
    $j(x) = x$. Es decir, $j$ es la función identidad.
\end{proof}

A partir de ahora
se considerarán solamente inmersiones elementales que no sean la identidad
entre modelos internos de ZFC.
Esto permite dar un nombre al $\delta$ de \ref{teo:elem-embed-trivial}.

\begin{defi}
    Sea $j\colon M\embed N$ una inmersión elemental. El \concept{punto crítico} de $j$
    es el menor ordinal $\alpha$ tal que $j(\alpha)>\alpha$.
\end{defi}

El siguiente es un resultado que será de importancia para la tercera
demostración del Teorema de Kunen, en el capítulo \ref{cap:teorema-kunen}.

\section{Ultrapotencias}
Sea $I$ un conjunto no vacío, $U$ un ultrafiltro sobre $I$ y, para cada $i\in I$,
sean $A_i$ conjuntos no vacíos. Dadas dos funciones $f$ y $g$ pertenecientes al producto
cartesiano de los $A_i$, se define la relación de \concept{U-equivalencia}:
\[
    f=_U g \quad\text{si, y solo si,}\quad \set{i\in I\mid f(i)=g(i)}\in U.
\]

La relación anterior es una relación de equivalencia \autocite[Proposición 4.1.5]{chang_model_2012},
por lo que podemos considerar la clase de equivalencia de una función dada $f$:
\[
    f_U = \set{g\in \prod_{i\in I} A_i\mid g=_U f},
\]
el \concept{ultraproducto} de los $A_i$ se define como el conjunto de todas las $f_U$,
y lo denotamos por $\prod_U A_i$. En el caso de que los $A_i$ sean todos iguales, digamos que
a un conjunto $A$, el ultraproducto se conoce como \concept{ultrapotencia} y se denota, naturalmente,
por $\prod_U A$.

Si en la construcción anterior, para cada $i\in I$, se consideran modelos $\model{A}_i$ entonces
se puede construir un modelo $\prod_U \model{A}_i$, al que llamaremos igualmente
ultraproducto o ultrapotencia según sea el caso,
haciendo de $\prod_U A_i$ el universo del modelo y dando una
interpretación apropiada a las relaciones,
funciones n-arias y las constantes \autocite[Definición 4.1.6]{chang_model_2012}, donde lo importante es que
dicho modelo está bien definido \autocite[Proposición 4.1.7]{chang_model_2012}. Conviene, sin embargo, enunciar el teorema fundamental de los
ultraproductos, pues da la forma en la que podemos interpretar la satisfacción de fórmulas
en estas estructuras.
\begin{teo}\label{teo:ultraprod-fundamental}
    Sea $\prod_U \model{A_i}$ un ultraproducto e $I$ su conjunto de índices.
    Dada cualquier fórmula $\phi(x_1,\dots,x_n)$ del lenguaje y $(f_1)_U,\dots,(f_n)_U\in\prod_{i\in I} \model{A}_i$,
    \[
        \prod_U \model{A}_i\models\phi((f_1)_U,\dots,(f_n)_U)
        \quad\text{si, y solo si,}\quad
        \set{i\in I\mid \model{A}_i\models\phi(f_1(i),\dots,f_n(i))}\in U.
    \]
\end{teo}

Como es usual para esta clase de teoremas en la teoría de modelos,
el resultado anterior se demuestra haciendo inducción sobre la complejidad
de $\phi$ \autocite[Teorema 4.1.9]{chang_model_2012}.
El teorema \ref{teo:ultraprod-fundamental} tiene varios corolarios importantes, de mayor
utilidad será el hecho de que existe una inmersión elemental
$j\colon \model{A}\embed\prod_U\model{A}$ \autocite[Corolario 4.1.13]{chang_model_2012},
llamada normalmente inmersión canónica.
En efecto, se define $j(\alpha)$
para $\alpha\in\model{A}$ como la clase de equivalencia de la función constantemente
igual a $\alpha$.

La primera dificultad para extender el concepto de ultrapotencia al universo
proviene de que, dada $f\colon I\to V$ y $U$ ultrafiltro sobre $I$,
la clase de equivalencia $f_U$ como se definió anteriormente es una clase propia.
Esto motiva un pequeño ajuste a la definición:
\[
    \ecl{f} = \set{g \in f_U\mid \forall h\; (h\in f_U\implies \rank(g)\leq\rank(h))},
\]
es decir, por la clase de $f$ se entiende ahora el conjunto de las funciones
de $f_U$ con rango mínimo. Entonces, si $\alpha$ es el ordinal más pequeño para
el que existe una función de rango $\alpha$ en $\ecl{f}$ esta clase de equivalencia estará
contenida en $V_{\alpha+1}$ y será, por tanto, un conjunto. Se puede entonces definir
el universo del modelo de ultraproducto que se busca como el conjunto de todas las $\ecl{f}$.
Si a este universo le añadimos la relación $\Eu$ dada por:
\[
    \ecl{g} \Eu \ecl{f}
    \quad\text{si, y solo si,}\quad
    \set{i\in I\mid g(i)\in f(i)}\in U,
\]
se obtiene un modelo
denotado por $\Ult(V,U)$.

Vale la pena destacar que \ref{teo:ultraprod-fundamental} sigue aplicando para
$\Ult(V,U)$ con la acotación de que, puesto que ahora está involucrada la relación
de satisfacción para clases propias, debe ser interpretado como un esquema infinito de teoremas.

La última herramienta teórica relacionada con ultrapotencias que será necesaria viene
dada por los siguientes dos teoremas. Primero, una condición extra sobre $U$ da como resultado
modelos bien fundados y, además, la relación $\Eu$ es tipo-conjunto.

\begin{teo}\label{teo:Eu-well-founded}
    Si $U$ es $\omega_1$-completo entonces $\Eu$ es una relación bien fundada.
\end{teo}
\begin{proof}
    Para la implicación directa, sea $\seq{\ecl{f_n}\mid n<\omega}$ tal que
    $\ecl{f_{n+1}}\Eu\ecl{f_n}$ para $n<\omega$,
    entonces $\bigcap_n\set{i\in I\mid f_{n+1}(i)\in f_n(i)}\neq\emptyset$
    da una sucesión infinita descendiente de $\in$.

    Ahora, sean $\set{X_n\mid n\in\omega}$ subconjuntos de $U$ tales que
    $\bigcap_{n<\omega} X_n\notin U$ entonces se definen $g_k\colon I\to V$
    para $k<\omega$ de la siguiente forma:
    \[
        g_k(i) =
        \begin{cases}
          n-k\quad & \text{si $i\in(\bigcap_{m<n} X_m) - X_n$ y $n\geq k$,}\\
          0\quad & \text{en cualquier otro caso.}
        \end{cases}
    \]
    Entonces,
    \[
        \set{i\in S\mid g_{k+1}(i)\in g_k(i)}
        \supseteq
        \bigcap_{m\leq k} X_m - \bigcap_{n\in\omega} X_n
        \in U,
    \]
    para $k\in\omega$ y la sucesión $\seq{(g_n)_U^0\mid n\in\omega}$ es testigo de que $\Eu$
    no está bien fundada.
\end{proof}

\begin{teo}\label{teo:Eu-set-like}
    La relación $\Eu$ es tipo-conjunto.
\end{teo}
\begin{proof}
    \def\Ou{^0_U}
    Sean $(g)\Ou, (f)\Ou\in \Ult(U,V)$ tales que $(g)\Ou \Eu (f)\Ou$ y $g_0\in(g)\Ou$.
    Se define $g_1\colon S\to V$ mediante:
    \[
        g_1(i) =
        \begin{cases}
          g_0(i)\quad & \text{si $g_0(i)\in f(i)$,}\\
          0\quad & \text{en cualquier otro caso.}
        \end{cases}
    \]
    Entonces $g_1\in (g)\Ou$ y $\rank(g_1)\leq\rank(f)$. Luego, $\rank((g)\Ou)\leq\rank(f)+1$,
    y se tiene que $\set{(g)\Ou\mid (g)\Ou\Eu (f)\Ou}\subseteq V_{\rank(f)+2}$ es un conjunto.
\end{proof}

Se sigue de los teoremas \ref{teo:Eu-well-founded} y \ref{teo:Eu-set-like},
usando el teorema de colapso de Mostowski, que si $U$ es $\omega_1$-completo
existe una clase transitiva $M_U$ y un isomorfismo $\pi_U$ tales que:
\[
    \pi_U\colon\Ult(V,U)\to\op{M_U,\in},
\]
además, debido a \ref{teo:ultraprod-fundamental}, $M_U$ es un modelo interno
de ZFC.

A partir de ahora, se utilizará la notación $[f]_U$ para $\pi_U((f)_U^0)$
con $f\colon S\to V$. En algunos casos, el ultrafiltro $U$ será claro dado el contexto
y se prescindirá del subíndice. Sea $f_x$ la función constantemente igual a $x$ y,
recordando el hecho de que existe una inmersión elemental $j$ de un modelo en su ultraproducto,
sea $j_U\colon V\embed M_U$ definida, para $x\in V$, por:
\[
    j_U(x) = [f_x]_U.
\]
La función $j_U$ es una inmersión de $V$ en $M_U$, debido a \ref{teo:ultraprod-fundamental}.
Lo anterior se resumirá de la siguiente forma:
\[
    j_U\colon V\embed M_U\approxeq\Ult(U,V),
\]
donde los subíndices se omitirán siempre que el ultrafiltro $U$ quede claro
del contexto.

Ahora que se tiene a disposición el modelo $\Ult(U,V)$ y su colapso transitivo,
diversos resultados de la teoría de cardinales medibles pueden ser establecidos.
De estos, quizás el de mayor importancia es aquel debido a Scott: si existe
un cardinal medible entonces $V\neq L$. La demostración de este hecho se
escapa del objetivo de este texto, sin embargo, hace falta demostrar un
último resultado referente a ultrapotencias que se usará más adelante.

\begin{teo}\label{teo:ultrafilters-omega1}
    Sea $U$ un ultrafiltro $\omega_1$-completo sobre un conjunto $S$ y
    $j\colon V\embed M\approxeq\Ult(U,V)$. Entonces,
    \begin{enumerate}[label=(\roman*)]
        \item Sea $X$ tal que $j"X\in M$ y $Y\subseteq M$ para el cual $|Y|\leq|X|$,
            entonces $Y\in M$.
        \item Para cualquier ordinal $\gamma$, $j"\gamma\in M$ si, y solo si,
            $\hbox{}^\gamma\kern-1pt M\subseteq M$.
        \item $j"(|S|^+)\notin M$.
    \end{enumerate}
\end{teo}

\begin{proof}\phantom{a}
    \begin{enumerate}[label=(\roman*)]
        \item Interpretemos a $Y$ como $\set{[f_x]\mid x\in X}$. Puesto que $j"X\in M$,
            existe $h\colon S\to\pwset{X}$ tal que $[h]=j"X$. Se define $g\colon S\to V$
            haciendo que $g(i)$ sea la función con dominio $h(i)$ que satisface
            $g(i)(x) = f_x(i)$. Entonces $[g](j(x)) = [f_x]$ para cada $x\in X$ y
            $\ran([g]) = Y$.
        \item Esta parte se sigue de la parte anterior.
        \item Sea $[f]\in M$. Si $A = \set{i\in S\mid |f(i)|\leq|S|}\in U$,
            entonces existe $\alpha\in|S|^+ - \bigcup\set{f(i)\mid i\in A}$ tal que
            $j(\alpha)\notin [f]$. De lo contrario, $B=\set{i\in S\mid |f(i)|>|S|}\in U$
            y existe una función inyectiva $h$ en $B$ que satisface $h(i)\in f(i)$
            para cada $i\in B$, y entonces $[h]\in[f]-j"V$. En cualquiera de los dos casos,
            $[f]\neq j"(|S|^+)$.
    \end{enumerate}
\end{proof}

\fi
% ------------------------------------------------------------------------------
% Cap 2: Teorema de Kunen.
% ------------------------------------------------------------------------------
\ifmainproofs
\chapter{El teorema de inconsistencia de Kunen}
\label{cap:teorema-kunen}
\thispagestyle{empty}

En la introducción se hizo alusión a la jerarquía de los cardinales grandes,
comenzando por los distintos tipos de inaccesibilidad hasta los diversos
grados de compacidad. El siguiente teorema, mencionado ya numerosas veces a lo largo
del texto, delimita la jerarquía de cardinales grandes en ZFC.

\begin{teo}[Kunen]\label{teo:kunen}
    Si $j\colon V\embed M$, entonces $M\neq V$.
\end{teo}

\section{Demostración de Kunen}

La primera demostración del teorema que se verá es aquella
del propio Kunen \autocite{kunen_elementary_1971}, adaptada
por Kanamori \autocite{kanamori_higher_2009}. Pero primero,
hace falta establecer un resultado, debido a Erdös-Hajnal \autocite{erdos_problem_1966},
sobre funciones $\omega$-Jónsson.

\begin{defi}
    Sea $x$ un conjunto de ordinales, $[x]^\omega = \set{y\subseteq x\mid \text{$y$ es de orden $\omega$}}$
    y $f$ una función,
    $f$ es $\omega$-Jónsson para $x$ si, y solo si,
    $f\colon [x]^\omega\to x$ y para cualquier $y\subseteq x$
    tal que $|y|=|x|$ se tiene $f"[y]^\omega = x$.
\end{defi}

\begin{teo}\label{teo:jonsson}
    Sea $\lambda$ un cardinal infinito, entonces existe una función $\omega$-Jónsson para $\lambda$.
\end{teo}

\begin{proof}
    Se demostrará el caso particular en el que $\lambda$ es un cardinal límite de cofinalidad $\omega$,
    para el caso general véase \autocite[Teorema 23.13]{kanamori_higher_2009}.
    Sea $\set{\op{x_\alpha,\gamma_\alpha}\mid \alpha < 2^\lambda}$ una enumeración del conjunto $[\lambda]^\lambda\times\lambda$.
    Para $\alpha<\lambda$ se puede escoger $s_\alpha\in[x_\alpha]^\omega$ tal que $s_\alpha\neq s_\beta$ para $\beta<\alpha$,
    debido a que $2^\lambda=\lambda^{\aleph_0}$. Entonces cualquier $f\colon[\lambda]^\omega\to\lambda$
    tal que $f(s_\alpha)=\gamma_\alpha$ es $\omega$-Jónsson para $\lambda$.
\end{proof}

\begin{proof}[Primera demostración de \ref{teo:kunen}]
    Sea $\kappa=\crit{j}$ y $j^n$ la $n$-ésima iteración de $j$: para $x\in V$, $j^0(x)=x$
    y $j^{n+1}(x) = j(j^{n}(x))$. Sea $\lambda=\sup(\set{j^n(\kappa)\mid n\in\omega})$,
    nótese que $j(\lambda)=\lambda$ pues $j(\set{j^n(\kappa)\mid n<\omega}) = \set{j^n(\kappa)\mid 1\leq n<\omega}$.
    Además, como $\kappa$ es medible, en particular es inaccesible, y la elementalidad de $j$ implica que $j^n(\kappa)$
    es inaccesible para todo $n\in\omega$ y $2^\lambda = \lambda^{\aleph_0}$, se puede aplicar entonces
    el caso especial que se demostró de \ref{teo:jonsson}.
    Para obtener que $V\neq M$ basta con establecer $j"\lambda\notin M$.

    En busca de una contradicción, sea $j"\lambda\in M$ y $f$ una función $\omega$-Jónsson para $\lambda$.
    En $M$, $j(f)$ es $\omega$-Jónsson para $j(\lambda)=\lambda$ y $j"\lambda\in[\lambda]^\lambda\cap M$.
    Sea $s\in[j"\lambda]^\omega$. Existe entonces un $t\in[\lambda]^\omega$ tal que $j(t)=j"t=s$.
    Se sigue que $j(f)(s) = j(f)j(s) = j(f)j(t) = j(f(t))\in j"\lambda$, pero esto implica
    \[
        \lambda = j(f)[j"\lambda]^\omega \subseteq j"\lambda
    \]
    que es imposible pues $\kappa\in\lambda- j"\lambda$.
\end{proof}

\section{Demostración de Woodin}

La siguiente demostración, debida a Hugh Woodin, apareció junto a otras en
la década de 1980. La existencia de particiones de conjuntos estacionarios
bajo las condiciones de \ref{teo:solovay} es el hecho clave para la demostración.

\begin{proof}[Segunda Demostración de \ref{teo:kunen}]
	Sea $\kappa=\crit{j}$ y $\lambda=\sup(\set{j^n(\kappa)\mid n\in\omega})$ igual que antes.
	El teorema \ref{teo:solovay} implica que existe $S\colon\kappa\to\pwset{\lambda^+}$ tal que
	$\ran(S)$ es una partición de $\set{\xi < \lambda^+\mid \cf(\xi) = \omega}$ en subconjuntos
	estacionarios en $\lambda^+$.

	Puesto que $j(\lambda)=\lambda$, se tiene $\lambda^+\leq j(\lambda^+) = {\lambda^+}^M\leq \lambda^+$
	y prevalece la igualdad. Por la elementalidad de $j$, $j(S)\colon j(\kappa)\to\pwset{\lambda^+}$ y
	\[
		(
		    j(S)(\kappa)\subseteq\set{\xi < \lambda^+\mid \cf(\xi) = \omega}
		    \,\text{es estacionario en $\lambda^+$}
		)^M.
	\]

	Si $M=V$, entonces $j(S)(\kappa)$ es estacionario en $\lambda^+$ visto en el universo y,
	debido a la $\lambda^+$-completitud, $j(S)(\kappa)\cap S(\alpha_0)$ es estacionario en $\lambda^+$
	para algún $\alpha_0<\kappa$. El conjunto
	\[
	    C = \set{\xi<\lambda^+\mid j(\xi)=\xi \land \cf(\xi)=\omega}
	\]
	es $\omega$-cerrado y no acotado en $\lambda^+$. Se sigue de \ref{teo:stationary-intersect} que
	existe un $\xi_0\in (j(S)(\kappa)\cap S(\alpha_0))\cap C$. Pero,
	$\xi_0 = j(\xi_0)\in j(S(\alpha_0)) = j(S)(\alpha_0)$ y entonces $\xi_0\in j(S)(\kappa)\cap j(S)(\alpha_0)$.
	Esto último es imposible, pues la elementalidad de $j$ implica que $j(S)$ consiste únicamente de conjuntos
	disjuntos dos a dos.
\end{proof}

\section{Demostración de Harada}

La última demostración que se dará del teorema de Kunen es una versión simplificada
por Kanamori~\autocite{kanamori_higher_2009} de una demostración de Mikio Harada.
Antes de poder entrar de lleno en dicha demostración, hará falta desarrollar
un poco la teoría de filtros normales.

Es necesario, en primer lugar, redefinir la noción de intersección diagonal.
Sea $\kappa$ un cardinal arbitrario, $S$ un conjunto y $F$ un filtro sobre $\pwset[\kappa]{S}$.
Si $\seq{X_i\mid i\in S}\in {}^S\pwset{\pwset S}$ entonces su intersección diagonal viene
dada por $\set{x\in\pwset{S}\mid x\in\bigcap_{i\in x} X_i}$ y se denota por $\dint_{i\in S} X_i$.

\begin{defi}
    Sean $\kappa, F$ y $S$ igual que en el párrafo anterior.
    $F$ es fino si, y solo si, $F$ es $\kappa$-completo y, para cualquier $i\in S$,
    $\set{x\in\pwset[\kappa]{S}\mid i\in x}\in F$.
    $F$ es normal si, y solo si, $F$ es fino y, para cualquier sucesión
    $\seq{X_i\mid i\in S}\in {}^SF$, $\dint_{i\in S} X_i\in F$.
\end{defi}

La definición anterior es equivalente \autocite[Ejercicio 22.5]{kanamori_higher_2009}
a la definición usual de filtro normal, esta es, aquella donde no se añade
la caracterización de finura \autocite[52]{kanamori_higher_2009}.
En vista de este hecho, se tiene el siguiente teorema.

\begin{teo}\label{teo:exists-k}
    Sea $j\colon V\embed M$ y $\kappa=\crit{j}$. Se define un ultrafiltro $U$
    de la siguiente forma:
    \[
        X\in U\quad\text{si, y solo si,}\quad X\subseteq \kappa \,\land\, j"\kappa\in j(X),
    \]
    Entonces existe una inmersión elemental $k\colon M_U\embed M$ tal que $k\circ j_U = j$.
\end{teo}

\begin{proof}
    Sea $k\colon M_U\embed M$ dada por $k([f]_U)=j(f)(\kappa)$. Entonces, para
    cualquier fórmula $\phi(v_1,\dots,v_n)$
    \begin{align*}
        M_U \models\phi[[f_1]_U,\dots,[f_n]_U]
        &\quad\text{si, y solo si,}\quad
        \set{\xi<\kappa\mid\phi[f_1(\xi),\dots,f_n(\xi)]} \in U \\
        &\quad\text{si, y solo si,}\quad
        M \models \phi[j(f_1)(k),\dots,j(f_n)(\kappa)]
    \end{align*}
    por la definción de $U$. Entonces, $k$ es elemental y es claro que $k\circ j_U=j$.
\end{proof}

La siguiente es una adaptación del concepto de conjunto estacionario al
contexto actual.

\begin{defi}
    Sea $F$ un filtro sobre un conjunto $I$, $X\subseteq I$ es $F$-estacionario si, y solo si,
    $X\cap Z\neq\emptyset$ para todo $Z\in F$.
\end{defi}

Para los filtros normales se tiene un resultado similar a \ref{teo:fodor}.

\begin{teo}\label{teo:normal-fodor}
    Un filtro $F$ sobre $\pwset[\kappa]{S}$ es normal si, y solo si,
    siempre que $X$ sea $F$-estacionario y $f$ una función de elección para $X$,
    existe un $s\in S$ tal que $f^{-1}(\set{s})$ es $F$-estacionario.
\end{teo}

\begin{proof}
    Para la primera implicación, supongamos que $f^{-1}(\set{s})$ no es $F$-estacionario
    para ningún $s\in S$ de tal forma que $\pwset[\kappa]{S}-f^{-1}(\set{s}) \in F$.
    Entonces $\dint_{s\in S}\pwset[\kappa]{S}-f^{-1}(\set{s})\in F$ por normalidad
    y $X\cap \dint_{s\in S}\pwset[\kappa]{S}-f^{-1}(\set{s})\neq\emptyset$
    pues $X$ es $F$-estacionario. Lo último implica que existe
    $x\in X\cap \dint_{s\in S}\pwset[\kappa]{S}-f^{-1}(\set{s})$
    pero entonces $f(x)\neq s$ para todo $s\in x$ o,
    $f(x)\notin x$, lo cual es imposible, pues $f$ es función de elección para $X$.

    En el sentido contrario, si existe $\seq{X_i\mid i\in S}\in {}^SF$ tal que
    $\dint_{i\in S} X_i \notin F$ entonces $\pwset[\kappa]{S} - \dint_{i\in S} X_i$
    es $F$-estacionario. Definimos una función $g$ sobre este conjunto de la siguiente
    forma: $g(x)$ es el menor $s$ tal que $s\notin X_s$. Se sigue que $g$ es función
    de elección para $X$ y $g^{-1}({s})\cap X_s=\emptyset$ para todo $s\in S$, que es
    imposible, pues $g^{-1}({s})$ es $F$-estacionario por hipótesis.
\end{proof}

Es posible construir inmersiones partiendo de ultrafiltros normales,
cuyas propiedades serán centrales para el argumento de Harada que se dará más adelante.
Sea $\kappa\leq\gamma$ y $U$ un ultrafiltro normal sobre $\pwset[\kappa]{\gamma}$.
Debido a que esto implica $\kappa > \omega$, $U$ es $\omega_1$-completo y podemos tomar
la inmersión canónica $j_U\colon V\embed M_U\approxeq\Ult(V,U)$.
Sea $\id\colon\pwset[\kappa]{\gamma}\to\pwset[\kappa]{\gamma}$ la función identidad, entonces:
\begin{enumerate}[label=(\roman*)]
    \item $[\id]_U = j_U"\gamma$ y ${}^\gamma M_U\subset M_U$.
    \item $\crit{j_U} = \kappa$ y $\gamma < j_U(\kappa)$.
\end{enumerate}

Para (\textsc{i}), $[\id]_U \subseteq j_U"\gamma$ se sigue de la finura de $U$ y
\ref{teo:normal-fodor} da la otra contención. El hecho de que ${}^\gamma M_U\subset M_U$
se sigue ahora de \ref{teo:ultrafilters-omega1}. Para (\textsc{ii}), $\crit{j_U}\geq\kappa$
es consecuencia de la $\kappa$-completitud. También, $\ot([\id]_U) < j_U(\kappa)$ pertenece
a $M_U$ debido a que $\set{x\subset\gamma\mid\ot(x)<\kappa}\in U$ y
$\gamma=\ot(j_U"\gamma)=\ot([\id]_U)$.

\begin{teo}\label{teo:ultrafilter}
    Sea $\kappa<\gamma$, $U$ ultrafiltro sobre $\pwset[\kappa]{\gamma}$,
    $j\colon V\embed M\approxeq\Ult(V, U)$ y $|\gamma|=\lambda$. Entonces,
    para $\alpha\leq\gamma$,
    \[
        j"\alpha=[\seq{x\cap \alpha\mid x\in\pwset[\kappa]{\lambda}}]_U
        \quad\text{y}\quad
        \alpha = [\seq{\ot(x\cap \alpha)\mid x\in\pwset[\kappa]{\lambda}}]_U.
    \]
\end{teo}

\begin{proof}
La normalidad de $U$, como se mostró antes, implica $j"\gamma = [\id]_U$.
Nótese ahora que, para $\alpha\leq\gamma$, $j"\alpha=j"\gamma\cap j(\alpha)$
y $\alpha=\ot(j"\alpha)$.
\end{proof}

\begin{proof}[Tercera Demostración de \ref{teo:kunen}]
    Sea $\kappa=\crit{j}$, $j^n$ la $n$-ésima iteración de $j$,
    y $\lambda=\sup(\set{j^n(\kappa)\mid n\in\omega})$; igual que antes.
    Asumamos, en busca de una contradicción, que $j"\lambda\in M$.
    Nótese que dada cualquier biyección $g\colon 2^\lambda\to\pwset{\lambda}$ se tiene que
    $j"\lambda\in\pwset{\lambda}^M = \ran(j(g))$ y se puede considerar el ordinal $\sigma$ más
    pequeño tal que existe una función inyectiva $F\colon\sigma\to\pwset{\lambda}$ para la cual
    $j"\lambda\in \ran(j(F))$. Tomemos una $F$ como la anterior y llamemos $S=\ran(F)$.
    Se define ahora un ultrafiltro $U$ de la siguiente forma:
    \[
        X\in U\quad\text{si, y solo si,}\quad X\subseteq S \,\land\, j"\lambda\in j(X).
    \]
    Tal como está definido, $U$ es $\omega_1$-completo y se puede tomar la inmersión canónica:
    \[
        i\colon V\embed N\approxeq\Ult(V, U).
    \]

    Es ahora que la teoría de ultrafiltros normales cobra relevancia: si $\id$
    es la identidad en $S$, la definción de $U$ implica, sucesivamente:
    \begin{enumerate}[label=(\roman*)]
        \item $i"\lambda = [\id]_U$ por lo que $i"\lambda\in\ran(i(F))$,
        \item $\alpha = [\seq{\ot(x\cap\alpha)\mid x\in S}]_U$ para $\alpha<\lambda$ y
        \item $i\rest(\lambda+1) = j\rest(\lambda+1)$,
    \end{enumerate}
    donde (\textsc{i}) y (\textsc{ii}) se siguen de \ref{teo:ultrafilter} y la discusión
    posterior. Para (\textsc{iii}), se define $k\colon N\to M$ como
    $k([j]_U) = j(f)(j"\lambda)$ y entonces $k$ es elemental y $k\circ i = j$
    por \ref{teo:exists-k}; además, $k(\alpha)=\alpha$ para $\alpha<\lambda$ por
    (\textsc{ii}).

    La introducción de la ultrapotencia $\Ult(V, U)$ proporciona la estructura necesaria:
    debido a \ref{teo:ultrafilters-omega1}(\textsc{ii}), $i"\lambda\in N$ implica que
    $\pwset{\lambda}^N = \pwset{\lambda}$ y también $i\rest V_\lambda\in N$ puesto que
    $\lambda$ es un cardinal límite fuerte. Ahora para cualquier $X\in\pwset{\lambda}$,
    $i(X) = \bigcup_{\alpha<\lambda} i(X\cap\alpha)$ pues $i(\lambda)=\lambda$ y se sigue
    que $i"\pwset{\lambda} = \set{\bigcup_{\alpha<\lambda} i(X\cap\alpha)\mid
    X\in\pwset{\lambda}}\in N$. Apelando nuevamente a \ref{teo:ultrafilters-omega1}(\textsc{ii})
    se sigue que $N$ tiene todas las enumeraciones de $\pwset{\lambda}^N=\pwset{\lambda}$ y,
    por lo tanto,
    \begin{equation}\label{eq:2-lambda}
        2^\lambda = (2^\lambda)^N = i(2^\lambda).
    \end{equation}

    Ahora pueden usarse las propiedades de $\sigma$ para obtener una contradicción.
    Las definiciones dadas anteriormente implican que $\sigma$ es un cardinal y
    $|S|=\sigma\leq 2^\lambda$. \ref{teo:ultrafilters-omega1}(\textsc{iii}) implica
    $i"\sigma^+\notin N$, pero dado que $i"\pwset{\lambda}\in N$, se sigue de
    \ref{teo:ultrafilters-omega1}(\textsc{i})(\textsc{ii}) que $2^\lambda<\sigma^+$.
    Se tiene, por tanto, que $2^\lambda = \sigma$. Pero entonces
    $i(\sigma)=\sigma$ por \ref{eq:2-lambda} de tal forma que $i(\sigma)=\sup(i"\sigma)$.
    Puesto que $i"\lambda\in\ran(i(F))$ por (\textsc{i})
    se sigue que $i"\lambda\in\ran(i(F)\rest i(\alpha)) = \ran(i(F\rest\alpha))$ para algún
    $\alpha<\sigma$. Finalmente, $j"\lambda\in\ran(j(F\rest\alpha))$ por definición de $U$
    e $i$, lo cual contradice la minimalidad de $\sigma$.
\end{proof}

\section{Las tres demostraciones, contrapuestas}

Todos los argumentos expuestos hasta ahora para demostrar el teorema de Kunen:
el del propio Kunen, el de Woodin y el de Harada; tienen en común el uso,
de forma fundamental, del axioma de elección para obtener un buen orden
de $\pwset{\lambda}$. Un poco más de detalle sobre esto se dará en la próximo
capítulo, pero vale la pena mencionar la similitud más importante
que poseé dicho trío de razonamientos.

Las similitudes, en principio, terminan allí y las demostraciones tanto de Kunen
como de Woodin son en esencia distintas a la Harada.
En efecto, los primeros hacen uso de contingencias combinatorias
(las funciones $\omega$-Jónsson en el caso de Kunen,
las particiones de conjuntos estacionarios para Woodin)
para obtener la contradicción buscada, mientras que el segundo
apela a la estructura intrínseca de las inmersiones elementales.

En este sentido, el argumento de Harada es a la vez (y quizás, paradójicamente)
el más complejo y el más convincente, pues no usa casi ninguna construcción
ajena a los conceptos fundamentales que están en juego: inmersiones elementales
y ultrapotencias.

Sin embargo, no conviene descartar, en base a lo anteriormente dicho,
los razonamientos de Kunen ni de Woodin. Por su parte, el de Kunen poseé
la virtud de ser el más sencillo de los tres, en el sentido de poseer
la menor cantidad de requisitos teóricos. Por otro lado, el de Woodin
poseé importancia histórica pues ayudó a cimentar el resultado
de Kunen.

Además, e independientemente de las valoraciones aquí hechas,
cada demostración tiene valor matématico intrínseco en tanto
que ilustran vías distintas por las que se puede llegar al mismo
resultado: la delimitación de la empresa de los cardinales grandes
en la teoría de conjuntos.

\chapter{El axioma de elección}
\thispagestyle{empty}

\epigraph
{
    El axioma de elección es obviamente cierto;
    el principio de buena ordenación obviamente falso;
    y ¿quién podrá decir algo sobre el lema de Zorn?
}
{Jerry L. Bona. \autocite[150]{schechter_handbook_1997}}

El axioma de elección, que postula la existencia,
para cada conjunto, de una función de elección,
es de naturaleza distinta a los demás axiomas en ZFC.
En efecto, los otros axiomas establecen la existencia
de ciertos conjuntos y explican como se construyen dichos
conjuntos. El axioma de elección no da ningún tipo de información
sobre la función cuya existencia afirma.

Desde su concepción este axioma ha sido, y, en algunos casos,
sigue siendo, objeto de polémica y discusión entre los matemáticos.
Es cierto que con el tiempo las polémica han disminuído, y el último
axioma de la teoría de conjuntos es aceptado hoy en día por la mayoría
de los matemáticos.
Sus evidentes ventajas, como la buena ordenación de cualquier conjunto,
especialmente, de los numeros reales, traen consigo comportamientos
patológicos como la paradoja de Banach-Tarski.

En el caso de los cardinales grandes, como hemos visto ya,
el axioma de elección impone una cota superior, la menor a día de hoy,
a la jerarquía de los cardinales grandes. Es entonces natural preguntarse
como ha de lucir dicha jerarquía bajo la ausencia del axioma de elección,
dicho de otra forma, ¿sigue siendo el resultado de Kunen cierto en ZF?.

Esta última pregunta no tiene, por ahora, respuesta. No es tampoco
el objetivo de este texto darle una respuesta. En su lugar,
se explorarán algunas generalizaciones y reformulaciones del teorema
de Kunen.

\section{Generalizaciones del teorema de Kunen}

\fi
% ------------------------------------------------------------------------------
%                                 BACKMATTER.
% ------------------------------------------------------------------------------
\backmatter
% ------------------------------------------------------------------------------
% Referencias
% ------------------------------------------------------------------------------
\ifreferencias
\ChapterNoNumber{Referencias}
\note{Arreglar formato}
\printbibliography[heading=mybib]
\fi
\end{document}
