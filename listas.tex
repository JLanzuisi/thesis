\chapter*{Lista de Símbolos}

En la lista siguiente, $C$ es un conjunto.

\begin{center}
    \begin{tabular}{cl}
        Símbolo & Significado \\
        \hline\noalign{\smallskip}
        $\pwset{C}$ & Conjunto de partes.\\
        $\sup(C)$ & Supremo, es decir, $\bigcup C$.\\
    \end{tabular}
\end{center}

\newpage
\chapter*{Lista de Abreviaturas}

\begin{center}
    \begin{tabular}{cl}
        Abreviatura & Significado \\
        \hline\noalign{\smallskip}
        ZF & Teoría de conjuntos de Zermelo-Fraenkel.\\
        AC & Axioma de elección.\\
        ZFC & ZF al añadir AC.\\
        NBG & Teoría de conjuntos de Von Neumann, Bernays y Gödel.\\
        CH & Hipótesis del continuo: $2^{\aleph_0} = \aleph_1$.
    \end{tabular}
\end{center}
