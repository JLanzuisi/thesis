\documentclass[12pt]{article}

\begin{document}

Para el capítulo II de la tesis:

\begin{enumerate}
    \item Teorema 23.13
    \item Primera demostración del teorema de Kunen página 320, que usa
        las funciones $\omega$-Jónsson de 23.13.
    \item Segunda demostración del teorema de Kunen en la página 320, debida Woodin que usa
        el toerema que ya tengo sobre particiones de conjuntos estacionarios.
        Quizás sería bueno incluir la acotación que esta justo después de teorema,
        de que el mismo argumento se puede hacer sin usar el conjunto $C$.
    \item Tercera demostración del toerema de Kunen en la página 321, debida a Harada.
        Usa también el teorema 22.4 que ya tengo demostrado.
        Hace falta revisar la teoría de ultrafiltros normales (página 52 \S5, y página 298 \S22)
        para demostrar 22.12:
        \begin{enumerate}
            \item La definición de \emph{fino} y \emph{normal} en pp.301.
            \item Demostrar 22.6.
            \item Los puntos (i) y (ii) justo despues de 22.6.
            \item Demostrar 22.11.
            \item Demostrar 22.12.
        \end{enumerate}
\end{enumerate}

Para el capítulo III aún no tengo nada concreto.
\end{document}
