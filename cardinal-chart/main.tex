\documentclass[tikz]{standalone}

\usepackage{mathtools}

\usetikzlibrary{shapes.geometric, arrows, positioning, backgrounds}
\tikzstyle{node} =
    [
        rectangle,
        align=left,
        fill=black!10,
        fill opacity=0.9,
        text opacity=1,
        text width=10cm,
    ]
\tikzstyle{arrow} = [thick,->,>=stealth]
\tikzstyle{arrowtext} =
    [
        rectangle,
        font=\scriptsize,
        align=left,
        text opacity=1,
        midway,
        above,
    ]

\ExplSyntaxOn
\int_new:N \l__arrow_text_int

\NewDocumentCommand{ \defi }{ m m }
    {
        \centering\fontsize{13}{13}\selectfont #1.\par
        \normalfont\normalsize\raggedright
        #2
    }

\NewDocumentCommand{ \newnode }{ m m m m }
    {
        \node (#1) [node, #2]
            { \defi{#3} {#4} };
    }

\NewDocumentCommand{ \newcomment }{ m m }
    {
        \node () [node, #1]
            { #2 };
    }

\NewDocumentCommand{ \nimplies }{ o m m }
    {
        \IfNoValueTF { #1 }
            { \draw [arrow] (#2) -- (#3); }
            {
                \int_gincr:N \l__arrow_text_int
                \draw [arrow] (#2)
                    --
                    node [arrowtext] { ( \int_use:N \l__arrow_text_int ) }
                    (#3);
            }
    }
\ExplSyntaxOff

\begin{document}

\begin{tikzpicture}[node distance=3em]
    \newnode{weekinac}{}{Inaccesible (débil)}
        {
            Cardinal límite regular.
        }
    % \newcomment{left=of weekinac}
    %     {
    %         Cardinal grande mas pequeño.
    %     }
    \newnode{a-weekinac}{above=of weekinac}{$\alpha$-Inaccesible (débil)}
        {
            Recursivamente:
            \begin{itemize}
                \item $\kappa$ es $0$-débilmente inaccesible sii $\kappa$ es regular.
                \item $\kappa$ es $\alpha+1$-débilmente inaccesible sii $\kappa$ es
                    límite regular de cardinales $\alpha$-débilmente inaccesible.
                \item $\kappa$ es $\delta$-débilmente inaccesible sii $\kappa$ es
                    $\alpha$-débilmente inaccesible para todo $\alpha<\delta$.
            \end{itemize}
        }
    \newnode{weekmahlo}{above=of a-weekinac}{Mahlo (débil)}
        {
            Si el conjunto $\{ \rho < \kappa \mid \rho\;\text{es regular}\}$
            es estacionario en $\kappa$.
        }
    \newnode{inac}{right=of a-weekinac}{Inaccesible}
        {
            Cardinal regular y si $\lambda < \kappa$ entonces $2^\lambda < \kappa$.
        }
    \newnode{a-inac}{above=of inac}{$\alpha$-Inaccesible}
        {
            Recursivamente:
            \begin{itemize}
                \item $\kappa$ es $0$-débilmente inaccesible sii $\kappa$ es regular
                    y si $\lambda < \kappa$ entonces $2^\lambda < \kappa$.
                \item $\kappa$ es $\alpha+1$-débilmente inaccesible sii $\kappa$ es
                    límite regular de cardinales $\alpha$-débilmente inaccesible.
                \item $\kappa$ es $\delta$-débilmente inaccesible sii $\kappa$ es
                    $\alpha$-débilmente inaccesible para todo $\alpha<\delta$.
            \end{itemize}
        }
    \newnode{mahlo}{above=of a-inac}{Mahlo}
        {
            Si el conjunto $\{ \rho < \kappa \mid \rho\;\text{es inaccesible}\}$
            es estacionario en $\kappa$.
        }
    \newnode{a-mahlo}{above=of mahlo}{$\alpha$-Mahlo}
        {
            Recursivamente:
            \begin{itemize}
                \item $0$-Mahlo sii $\kappa$ es inaccesible.
                \item $(\alpha+1)$-Mahlo sii 
                    $\{\xi<\kappa\mid\text{$\xi$ es $\alpha$-Mahlo} \}$
                    es estacionario en $\kappa$.
                \item $\delta$-Mahlo sii es $\alpha$-Mahlo para cada
                    $\alpha < \delta$.
            \end{itemize}
        }
    \newnode{weekcompact}{above=of a-mahlo}{Compacto (débil)}
        {
            Si cualquier colección de sentencias $L_{\kappa\kappa}$ cumple con
            el teorema de compacidad debil: si toda subcolección de cardinalidad
            menor que $\kappa$, que use a lo sumo $\kappa$ símbolos no lógicos,
            tiene un modelo, entonces $L_{\kappa\kappa}$ también.
        }
    \newnode{measurable}{above=of weekcompact}{Medible}
        {
            Para $\kappa>\omega$; $\kappa$ es medible si existe una medida (real) $\kappa$-aditiva
            sobre $\kappa$.\\[1em]
            Equivalentemente:\\
            Si existe un ultrafiltro $\kappa$-completo sobre $\kappa$.
        }
    \newnode{compact}{above=of measurable}{Compacto}
        {
            Si cualquier colección de sentencias $L_{\kappa\kappa}$ cumple con
            el teorema de compacidad fuerte: si toda subcolección de cardinalidad
            menor que $\kappa$ tiene un modelo, entonces $L_{\kappa\kappa}$ también.
            \\[1em]
            Equivalentemente:
            \\
            Si para cualquier conjunto $S$, todo filtro $\kappa$-completo
            sobre $S$ se puede extender a un ultrafiltro $\kappa$-completo.
        }

\begin{scope}[on background layer]
    \nimplies{weekmahlo}{a-weekinac}
    \nimplies[Test2]{a-weekinac}{weekinac}
    \nimplies[Anohter test]{inac}{a-weekinac}
    \nimplies{a-inac}{inac}
    \nimplies{mahlo}{a-inac}
    \nimplies[Test]{mahlo}{weekmahlo}
    \nimplies{measurable}{weekcompact}
    \nimplies{compact}{measurable}
    \nimplies{weekcompact}{a-mahlo}
    \nimplies{a-mahlo}{mahlo}
\end{scope}
\end{tikzpicture}

\end{document}
