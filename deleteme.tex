\documentclass{article}

\usepackage{amsmath}

\DeclareMathOperator{\ran}{ran}
\DeclareMathOperator{\dom}{dom}

\usepackage{geometry}[left=.5cm, right=.5cm]

\begin{document}
    \noindent\fontsize{16}{25}\selectfont
    Si $i"\lambda\in\ran(i(F))$ entonces existe
    $\delta\in\dom(i(F)) = i(\sigma) = i(2^\lambda) = 2^\lambda$
    tal que $i(F)(\delta) = i"\lambda$. Entonces se puede tomar
    $\alpha$ en el intervalo
    $\delta< \alpha < 2^\lambda$ y $\delta\leq i(\delta)< i(\alpha)$
    implica $\delta\in i(\alpha)$ pues $\alpha$ es transitivo al ser un ordinal,
    y entonces $\delta$ es testigo de
    $i"\lambda\in\ran(i(F)|i(\alpha)) = \ran(i(F|\alpha))$.

    ¿Por qué $j"\lambda\in\ran(j(F|\alpha))$?
\end{document}
