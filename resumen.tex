\PrintUsbLogo
    {
        Universidad Simón Bolívar\\
        Decanato de estudios profesionales\\
        Coordinación de \coord
    }

\begin{center}
    \begin{minipage}{14cm}
        \centering
        \UppercaseBold
            {
                \MainTitle
            }
    \end{minipage}

    \vspace{.5cm}

    \UpperCase
        {
            Proyecto de grado
        } \\
    Realizado por: \autor \\
    Con la asesoría de: \tutor \\[.5cm]
    \UppercaseBold{Resúmen}

\end{center}

\note{El titulo menciona 3 demostraciones pero el resumen no dice cuales son.}
El teorema de inconsistencia de Kunen, que establece la inexistencia en
ZFC (teoría de conjuntos Zermelo-Fraenkel con el axioma de elección)
de cualquier inmersión elemental $j\colon V\prec V$ y, por lo tanto, de los cardinales
de Reinhardt, es un resultado central de la teoría de cardinales grandes.
El teorema de Kunen da una cota superior para la jerarquía de los cardinales grandes.

Se tratará el teorema mencionado a través de tres demostraciones, cada una de naturaleza
distinta. El libro de Kanamori \autocite{cohen_independence_1964} es la referencia estándar
en el estudio de los cardinales grandes y la fuente
de dichas demostraciones.

Para poder estudiar el resultado de Kunen en profundidad,
se divide el presente escrito en 3 capítulos más la introducción.
En la introducción se discuten los antecedentes históricos y la importancia
de este teorema.
El primer capítulo consta de nociones básicas necesarias para su enunciación y demostración.
El segundo capítulo introduce nociones también necesarias, pero ya no básicas, como los cardinales
supercompáctos o extendibles.
Para el tercer capítulo, se entra de lleno en el resultado de Kunen y sus demostraciones.

Las conclusiones discuten las posibles vías de investigación y desarrollo en esta área,
haciendo énfasis en un problema abierto asociado al teorema:
¿Seguirá siendo cierto el resultado de Kunen si se prescinde del axioma de elección?

\vspace{\fill}

\textbf{Palabras Clave:}
