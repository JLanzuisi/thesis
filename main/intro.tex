% Introducción:
%     - Conceptos básicos de teoría de conjuntos. Establecer la teoría de conjuntos en la que se va a trabajar y establecer notación.
% Citar CG de ivorra en el parrafo sobre la importancia.
% Mencionar la jerarquía de los CG, y explicar que se tratará una parte superior
% de la jerarquía: de los cardinales medibles para arriba. Quizás dar un breve resúmen
% de como es la jerarquía por debajo. (Los grafos que tiene Ivorra en CG podrían usarse)
\pagebreak
\chapter*{Introducción}

\epigraph
{
Del paraíso, que Cantor ha creado para nosotros, nadie ha de expulsarnos.
}
{David Hilbert. \autocite[pág 170]{hilbert_uber_1926}}

Las hipótesis de cardinales grandes son los axiomas matemáticos
más fuertes jamás postulados: estipulan la existencia de conjuntos
infinitos de tal tamaño, que no son decidibles en el marco de la teoría de conjuntos.
Los más pequeños entre ellos, siendo enormes, nunca son suficientemente
fuertes para demostrar la existencia de cardinales mayores.
El teorema de inconsistencia de Kunen es
una cota superior que impone el axioma de elección a las
hipótesis de cardinales grandes.
Pero, ¿qué utilidad pueden tener los cardinales grandes?
¿por qué interesa el resultado de Kunen?

La principal razón por la que el estudio de estos conjuntos infinitos
es relevante proviene del siguiente hecho:
es con ellos que cualquier aserción sobre consistencia relativa
puede medirse.
Sabemos gracias a Gödel que si en una teoría se puede desarrollar la aritmética
elemental, esta no puede demostrar su propia consistencia.
Si $T$ es una teoría de esta forma, lo que sí se puede es construir la teoría
$T'=T+\con{T},$
donde añadimos como nuevo axioma la consistencia
de $T$,
que evidentemente demuestra que $T$ es consistente; todo esto sin contradecir
el resultado de Gödel. Pero tenemos ahora un nuevo problema: la consistencia
de $T'$.
Consideremos entonces otra teoría,
\[T'' = T + \con{T} + \con{T + \con{T}},\]
que demuestra la consistencia de $T'$.
Podemos continuar de esta manera, definiendo $T''', T'''',\text{etc}$.
De esta manera se construye una jerarquía análoga a la de los ordinales,
en una torre ascendente infinita de consistencia relativa
\autocite[\S 7.7]{hamkins_lectures_2020}.

La conexión importante es la siguiente:
los cardinales grandes representan una instanciación de esta jerarquía.
Podemos entonces, a través de ellos, estudiar este universo infinito de
consistencia al que Gödel nos abrió las puertas.
Convendría hablar un poco más en detalle sobre los cardinales grandes
y su lugar en la tradición, empezada por Cantor, de la teoría de conjuntos.

\section*{Cardinales Grandes. Extensión hasta la Inconsistencia.}

Los cardinales grandes tienen sus orígenes en las investigaciones cantorianas
sobre conjuntos definibles de números reales y los números transfinitos.
Fue Felix Hausdorff \autocite{hausdorff_grundzuge_nodate} 
el primero en considerar un cardinal grande,
los débilmente inaccesibles.
Paul Mahlo \autocite{mahlo_uber_1911,mahlo_zur_1912,mahlo_zur_1913}
postulará después los cardinales que llevan su nombre.
Al considerar la clausura sobre la formación del conjunto de partes,
Sierpiński-Tarski \autocite{sierpinski_sur_1930} y  Zermelo \autocite{zermelo_uber_1930}
llegan a la noción de cardinal (fuertemente) inaccesible.

Stanisław Ulam \autocite{ulam_zur_1930}, al estudiar la medida de Lebesgue,
introduce los cardinales medibles y con ellos la primera
pregunta sobre la jerarquía de los cardinales grandes:
¿Es el primer cardinal inaccesible también medible?
El desarrollo de los cardinales grandes dependerá a partir de este
momento de la incorporación de la teoría de modelos (\cref{sec:models}) en las matemáticas.

La generalización de la lógica de primer orden, obtenida al
permitir una cantidad infinita de operaciones lógicas,
permitió a Tarski \autocite{tarski_problems_1966} definir los cardinales (débil y fuerte)
compactos como una generalización del teorema de compacidad
para estas lógicas. Los cardinales compactos dieron solución
a la pregunta propuesta unos párrafos más arriba:
el primer cardinal inaccesible no es medible.

El siguiente gran salto adelante vendría de la mano de
Paul Cohen \autocite{cohen_independence_1963,cohen_independence_1964}
y la invención del forcing como técnica
para establecer resultados de consistencia relativa.
Cohen usaría su nueva técnica para construir un modelo de la
teoría de conjuntos donde falla la hipótesis del continuo
y junto con un resultado anterior de Gödel---a saber, que
en el universo de los constructibles se verifica CH---%
logra resolver finalmente la gran pregunta de Cantor sobre cardinalidades
intermedias entre los naturales y el continuo.

Finalmente, en la década de 1970,
Solovay y Reinhardt comienzan a postular
hipótesis de cardinales grandes aún más fuertes
que las anteriores.
Al poner en el centro el concepto de inmersión elemental
(\cref{sec:elem-embed}),
nacen las nociones de cardinal supercompácto y extendible.

Reinhardt \autocite{reinhardt_ackermanns_1970},
generalizando su concepto de extendibilidad,
propone el mayor principio de reflección posible:
la existencia de una inmersión elemental $j\colon V\prec V$
y la consideración de $\crit{j}$ como cardinal grande.

Es aquí que irrumpe el resultado de Kunen,
estableciendo la imposibilidad de dicha inmersión
y delimitando por arriba la jerarquía de los cardinales
grandes. A partir de este momento, el desarrollo
de esta teoría se dará considerando cardinales más
débiles que el propuesto por Reinhardt,
para evitar la inconsistencia.

Al momento de demostrar su teorema, Kunen hace
uso del axioma de elección. Como se verá
más adelante, todas las demostraciones dadas aquí
dependerán del axioma de elección.
La pregunta natural es: ¿Realmente se necesita AC?
¿Es demostrable el teorema de Kunen en ZF?

Esta última pregunta es, actualmente, un problema abierto.
Lo que indica una posible vía por la que se puede desarrollar
el estudio del resultado de Kunen, y muestra de que más allá
de la potencia de dicho teorema, quedan aún preguntas por explorar.

\section*{Jerarquía de los cardinales grandes.}

\section*{Teoría de Conjuntos.}

De todas las axiomatizaciones posibles para la teoría de conjuntos,
se usará la de Zermelo-Fraenkel con el axioma de elección, tal como
aparece en cualquiera de las referencias estándar
\autocite{kunen_set_2013,jech_set_2003}. Quizás haga falta justificar un poco
esta elección, puesto que existen muchas teoría de conjuntos posibles \autocite{ivorra_teorias_2013}.

Discutir todas las teoría de conjuntos, desde la de Kaye-Foster hasta los nuevos fundamentos
de Quine, por ejemplo, sería una tarea para un tratado de lógica matemática.
Pero quizás si es conveniente justificar la elección de ZFC sobre NBG, que es
la otra teoría de uso común.

La principal diferencia entre ambas teorías es que NBG admite a las clases propias
como objetos dentro de la teoría. Aunque esta mayor generalidad pueda parecer una ventaja,
no lo será en lo que respecta al estudio del teorema de Kunen.
Es sabido que NBG es una extensión conservativa de ZF,
esto es, todo teorema de NBG que involucra solamente conjuntos es también demostrable en ZF
\autocite[pág. 70]{jech_set_2003}.
Se sigue que, siempre que consideremos conjuntos, las dos teorías son equivalentes.

En los casos en los que aparezcan clases propias, como cuando se consideran estructuras
cuyos dominios son clases, se restringirán de tal forma que el resultado sea formalizable
en ZFC: por ejemplo usando el hecho de que la relación de satisfacción $\models$ (\cref{sec:models})
es formalizable para clases cuando se restringe a las fórmulas $\Sigma_n$
\autocite[pág. 6]{kanamori_higher_2009}.

Tenemos entonces que NBG no ofrece ninguna ventaja tangible sobre ZFC
para el caso que se propone estudiar.
Esto aunado al hecho de que la tradición conjuntista ha considerado ZFC
como la teoría estándar, lleva a preferirla sobre otras.

Se asume entonces familiaridad con las nociones más básicas de la teoría de conjuntos
y de lógica de primer orden. 

