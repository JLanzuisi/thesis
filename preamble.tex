\documentclass
[
  12pt,
  letterpaper,
  openany,
  oneside,
  draft
]{book}

\ExplSyntaxOn
\makeatletter

\RequirePackage { geometry }
\RequirePackage { setspace }
\RequirePackage [ spanish ] { babel }
\RequirePackage [ tracking=smallcaps, babel=true, disable=ifdraft ] { microtype }
\RequirePackage { enumitem }
\RequirePackage { csquotes }
\RequirePackage { biblatex }
\RequirePackage { xcolor }
\RequirePackage { graphicx }
\RequirePackage { mathtools, amsthm }
\RequirePackage { dirtytalk }
\RequirePackage { epigraph }
\RequirePackage { hyperref }
\RequirePackage [ spanish ] { cleveref }
\RequirePackage { lineno }

% Cref
\crefname{section}{\S}{\S}

% Set page margins
\geometry
    {
        includehead,
        paper = letterpaper,
        left = 3cm,
        right = 2cm,
        bottom = 2cm,
        top = 2cm,
        marginparsep = 10pt,
        marginparwidth = 2cm,
    }

% Set fonts
\usepackage[T1]{fontenc}
\usepackage{mlmodern}
\usepackage[cal=boondoxo, calscaled=1.05, frak=euler]{mathalfa}
% Patch bold font
\renewcommand{\bfdefault}{b}
\DeclareRobustCommand\bfseries
        {\not@math@alphabet\bfseries\mathbf
         \fontfamily{lmr}\fontseries\bfdefault\selectfont}
\newcommand{\titlefont}{\bfseries\selectfont}

% Set basic spacing
\setlength { \parindent } { 8.46pt }
\setlength { \parskip } { 6pt }
\onehalfspacing

% Headers and footer
\pagestyle{headings}
\gdef\@oddhead{\hfil\thepage}

% Bibliography
\addbibresource{references.bib}

% Hyperref
\hypersetup
    {
        colorlinks = true,
        allcolors = black,
    }

% Theorems
\newtheoremstyle{defi}% hnamei
    {.5\baselineskip}% hSpace abovei
    {.5\baselineskip}% hSpace belowi
    {}% hBody fonti
    {}% hIndent amounti
    {\scshape}% hTheorem head fonti
    {.}% hPunctuation after theorem headi
    {.5em}% hSpace after theorem headi
    {}% hTheorem head spec (can be left empty, meaning ‘normal’)i
\theoremstyle{defi}
\newtheorem{defi}{Definición}[chapter]
\newtheoremstyle{lem}% hnamei
    {.5\baselineskip}% hSpace abovei
    {.5\baselineskip}% hSpace belowi
    {}% hBody fonti
    {}% hIndent amounti
    {\scshape}% hTheorem head fonti
    {.}% hPunctuation after theorem headi
    {.5em}% hSpace after theorem headi
    {}% hTheorem head spec (can be left empty, meaning ‘normal’)i
\theoremstyle{lem}
\newtheorem{lem}{Lema}[chapter]
\newtheoremstyle{teo}% hnamei
    {.5\baselineskip}% hSpace abovei
    {.5\baselineskip}% hSpace belowi
    {}% hBody fonti
    {}% hIndent amounti
    {\scshape}% hTheorem head fonti
    {.}% hPunctuation after theorem headi
    {.5em}% hSpace after theorem headi
    {}% hTheorem head spec (can be left empty, meaning ‘normal’)i
\theoremstyle{teo}
\newtheorem{teo}{Teorema}[chapter]

% Sectioning commands
\usepackage[explicit]{titlesec}
\titleformat
    {\chapter}
    [display]
    {\titlefont}
    {
        \centering
        \fontsize{14}{14}\selectfont
        \UpperCase{Capítulo}~\thechapter\\[10pt]
        \fontsize{12}{12}\selectfont
        \UpperCase{#1}
    }
    {0ex}
    {\centering}
\titleformat{name=\chapter, numberless}
    [block]
    {\titlefont}
    {\fontsize{14}{14}\selectfont\UpperCase{#1}}
    {0ex}
    {\centering}
\titleformat
    {\section}
    {\titlefont}
    {\thesection\fontsize{12}{12}\selectfont\ #1}
    {0ex}
    {}
\titleformat{name=\section, numberless}
    [block]
    {\titlefont}
    {\fontsize{12}{12}\selectfont #1}
    {0ex}
    {}

% USB logo
\cs_set_eq:NN \latex_centering:D \centering

\cs_new:Npn \th_print_logo_head:n #1
    {
        \group_begin:
            \latex_centering:D
            \includegraphics[scale=.3]{resources/usblogo} \\
            { \titlefont \UpperCase{#1} }
            \tex_par:D
        \group_end:
    }

\NewDocumentCommand{ \UpperCase }{ m }
    {
        \group_begin:
        \lsstyle
        \text_uppercase:n { #1 }
        \group_end:
    }

\NewDocumentCommand{ \UppercaseBold }{ m }
    {
        { \titlefont\UpperCase{#1} }
    }

\NewDocumentCommand{ \PrintUsbLogo }{ m }
    {
        \th_print_logo_head:n { #1 }
    }

\NewDocumentCommand{ \ToC }{}
    {
        \chapter*{Índice~General}
        \@starttoc{toc}
    }

% Notes
\reversemarginpar
\NewDocumentCommand{ \note }{ m }
{
    \@ifclasswith{ book } { draft }
    {
        \begingroup
            \color{blue}
            \sffamily
            \small
            \textlangle #1 \textrangle
        \endgroup
    } {}
    \PackageWarning{Notes}{Revisar~nota}
}

% Wrapping
\newlength{\wrapwd}
\NewDocumentCommand { \wrapto } { o o m }
{
    \IfNoValueTF { #1 }
        { \setlength{\wrapwd}{.6\linewidth} }
        { \setlength{\wrapwd}{#1} }
    \IfNoValueTF { #2 }
        { \let\wrapal\raggedright }
        { \let\wrapal#2 }
    \begin{minipage}{\wrapwd}
        \wrapal
        #3
    \end{minipage}
}

% Conditionals
\newif { \ifcaratula }
\newif { \ifpaginatitulo }
\newif { \ifresumen }
\newif { \ifdedicatoria }
\newif { \ifagradecimientos }
\newif { \iftoc }
\newif { \ifsimbolos }
\newif { \ifabreviaturas }
\newif { \ifintro }
\newif { \ifbasicos }
\newif { \ifreferencias }

\@ifclasswith{ book } { draft }
{
  \AddToHook{begindocument}
  {
    \newbox{\watermark}
    \savebox{\watermark}
    {
      \rotatebox{45}
      {
        \scalebox{7}
        {
          \textcolor{red!10}{BORRADOR}
        }
      }
    }
    \AddToHook{ shipout/background }
    {
      \put (\paperwidth/2 - \wd\watermark/2,-\paperheight/2 - \ht\watermark/2)
      {
        \usebox{\watermark}
      }
    }
  }
}{}

% macros
\newcommand { \MainTitle }
    {
        Estudio~comparativo~de~tres~demostraciones~
        del~teorema~de~inconsistencia~de~Kunen.
    }

\newcommand { \autor }
    {
        Jhonny~Lanzuisi~Berrizbeitia
    }

\newcommand { \tutor }
    {
        Jesús~Nieto~Martínez
    }

\newcommand { \coord }
    {
        Matemáticas
    }

\DeclareMathOperator{\Con}{Con}
\DeclareMathOperator{\Crit}{crit}

\NewDocumentCommand { \model } { m }
{
    \mathfrak { #1 }
}
\NewDocumentCommand { \set } { m }
{
    \left\{ #1 \right\}
}
\NewDocumentCommand { \lex } { m }
{
    \mathcal { #1 }
}
\NewDocumentCommand { \crit } { m }
{
    \Crit (\, #1\, )
}

\NewDocumentCommand { \pwset } { m }
{
    \mathcal{P}(#1)
}

\NewDocumentCommand{ \concept } { m }
{
    \emph{#1}
}

\NewDocumentCommand{ \op } { m }
{
     \langle #1 \rangle
}

\NewDocumentCommand{ \con } { m }
{
    \Con ( #1 )
}

\DeclareMathOperator*{ \dint } { \bigtriangleup }

\@ifclasswith{ book } { draft } { \linenumbers } {}

\DeclareDocumentCommand{ \LoadClass } { o m } {}

\makeatother
\ExplSyntaxOff
\dump
