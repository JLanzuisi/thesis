\documentclass[12pt]{article}

\usepackage{base}
\usepackage{draft}

\usepackage{lipsum}

\begin{document}
    \maketitle
    \thispagestyle{empty}

    \begin{abstract}
        A continuación se presentan los objetivos, metodología, justificación,
        cronograma e introducción en el marco del proyecto de grado del
        Br. Jhonny Alexander Lanzuisi Berrizbeitia, Carnet No. 15-10759, para su
        evaluación y posible aprobación por la Coordinación de Matemáticas de
        la USB. Cabe señalar que las posibles reformas que la misma dinámica
        de la investigación impongan o se susciten durante el desarrollo de
        este proyecto también serían presentadas oportunamente.
    \end{abstract}

    \section{Fases del proyecto}

    \begin{enumerate}[label=\Roman*.]
        \item Búsqueda y recolección de datos mediante la revisión
              bibliográfica y el establecimiento de conjeturas y conclusiones.
        \item Redacción del trabajo de grado.
    \end{enumerate}

    \section{Cronograma}

    \begin{description}
        \let\olditem\item
        \renewcommand\item[1][]{\olditem[\normalfont #1.]}
        \item[1er trimestre]
            \begin{minipage}[t]{\dimexpr\linewidth-2cm\relax}
                \begin{description}
                    \item[Semanas 1-4] Búsqueda y recolección de la bibliografía.
                    \item[Semanas 5-8] Revisión de la bibliografía planteada.
                    \item[Semanas 9-12] Redacción de la introducción y el primer capítulo
                                        del trabajo de grado.
                \end{description}
            \end{minipage}
        \item[2do trimestre]
            \begin{minipage}[t]{\linewidth}
                \begin{description}
                    \item[Semanas 1-12] Redacción de los capítulos restantes del trabajo de grado
                \end{description}
            \end{minipage}
        \item[3er trimestre]
            \begin{minipage}[t]{\linewidth}
                \begin{description}
                    \item[Semanas 1-6] Revisión del borrador y redacción final del
                                       proyecto de grado.
                    \item[Semanas 7-12] Presentación y defensa.
                \end{description}
            \end{minipage}
    \end{description}

    \section{Justificación}
    El proyecto de grado se justifica considerando
    su relación con un problema abierto actualmente en el área,
    que ataña a los cimientos de la teoría:
    \begin{quote}
        ¿Existe una inmersión elemental $j\colon V\to V$ en ZF? (Ver la definición \ref{defi:embed}).
    \end{quote}
    Se sabe, gracias al teorema de Kunen (ver teorema \ref{eq:kunen_inc} más abajo) propuesto como objeto de estudio,
    que la respuesta a dicha pregunta es negativa al añadir el axioma de elección. Esto dice
    que el teorema de inconsistencia representa la mayor restricción, respecto a cardinales grandes,
    que impone el axioma de elección sobre el universo \cite[Pág 324]{kanamori_higher_2009}.

    Queda entonces clara la importancia de este tema en el desarrollo de la Teoría de Conjuntos y,
    por lo tanto, la útilidad de su estudio.

    \section{Modalidad y metodología}
    Este proyecto es una investigación centrada en la Teoría de Conjuntos,
    específicamente sobre las hipótesis de cardinales grandes.
    Como tal es de tipo teórico y requiere fundamentalmente del estudio
    de referencias bibliográficas (libros, artículos, etc), algunas de las referencias principales son citadas
    al final de este texto.
    Lo anterior implica que el estudio comparativo propuesto es de naturaleza cualitativa,
    pues su reflexión solo es posible en términos teóricos.

    \section{Introducción}
    Se parte del sistema de axiomas ZFC, siguiendo la notación de \cite{kanamori_higher_2009}.
    Denotamos por $V_\alpha$, para un ordinal $\alpha$, al universo de Von Neumann; esto es,
    a los conjuntos cuyo rango es menor que $\alpha$. Denotamos también por $V$ al universo,
    es decir, a la clase de todos los conjuntos.
    Las primeras letras griegas $\alpha,\beta,\dots$ denotan ordinales mientras que las
    letras centrales $\kappa,\lambda,\mu,\dots$ representan cardinales infinitos.

    \begin{defi}[defi:embed]
    Si $\model{M_1} = \set{M_1,\dots}$ y $\model{M_2} = \set{M_2,\dots}$ son dos modelos de un
    lenguaje $\lex{L}$, una función inyectiva $j\colon M_1\to M_2$ es una inmersión elemental si,
    y solo si, para cualquier fórmula $\phi(v_1,\dots,v_n)$ de $\lex{L}$ y $x_1,\dots,x_n\in M_1$ se cumple
    \[
        \model{M_1}\models \phi(x_1,\dots,x_n) \leftrightarrow \model{M_2}\models \phi(j(x_1),\dots,j(x_n)).
    \]
    Dicha inmersión usualmente se denota $j\colon \model{M_1}\prec\model{M_2}$.
    \end{defi}

    \begin{defi}
    Sea $j$ una inmersión elemental. El punto crítico de $j$, denotado $\crit{j}$, es el ordinal $\delta$ más pequeño
    tal que $j(\delta)\neq\delta$; esto es, el primer ordinal al que $j$ mueve de su posición original.
    \end{defi}

    En su tesis de 1976, William Reinhardt \cite{reinhardt_ackermanns_1970} propuso una noción de extensión en
    la teoría de conjuntos de Ackermann \cite{ackermann_zur_1956} que se puede traducir a ZFC de la siguiente forma:
    \begin{defi}
        $\kappa$ es $\eta$-extensible si, y solo si, existe un $\zeta$ y una $j\colon V_{\kappa+\eta}\prec V_\zeta$
        tal que $\crit{j} = \kappa$ y $j(\kappa) > \eta$.
        $\kappa$ es extensible si es $\eta$-extensible para todo $\eta>0$.
    \end{defi}
    \begin{defi}
        $\kappa$ es extensible si es $\eta$-extensible para todo $\eta>0$.
    \end{defi}


    Al finalizar su tesis, Reinhardt considera el siguiente axioma como una expresión fuerte
    de la noción de extensión mencionada antes:
    \begin{prop}
        Existe una inmersión elemental $j\colon V\to V$.
    \end{prop}

    Kunen \cite{kunen_elementary_1971} demostraría después que dicha proposición es falsa
    en ZFC:
    \begin{teo}[eq:kunen_inc]
        Sea $j\colon V\prec M$. Entonces $M\neq V$.
    \end{teo}

    El teorema anterior parece cuantificar universalmente sobre todas las inmersiones $j$,
    lo cual no es formalizable es ZFC pues dicha colección es una clase propia. De aquí se sigue
    que el teorema \ref{eq:kunen_inc} es un esquema infinito de teoremas: uno por cada $j$ que se tome.

    Existen diversas demostraciones para el teorema \ref{eq:kunen_inc}, de las cuales se considerarán tres:
    \begin{enumerate}
        \item La demostración original dada por Kunen \cite{kunen_elementary_1971}, que
              es de naturaleza combinatoria y se fundamenta en la noción de función
              $\omega$-Jónsson, debida a Erdös-Hajnal \cite{erdos_problem_1966}.
        \item Una demostración de Woodin \cite[Pág 320]{kanamori_higher_2009}, también de naturaleza combinatoria
              pero haciendo énfasis en particiones de conjuntos estacionarios.
        \item Finalmente, una demostración de Mikio Harada \cite[Pág 321]{kanamori_higher_2009}.
    \end{enumerate}


    Del teorema \ref{eq:kunen_inc} se sigue una pregunta natural, a la que se aludió antes, ¿seguirá siendo
    cierto el resultado de Kunen al prescindir del axioma de elección? 

    A modo de conclusión se consideran los avances recientes relacionados al problema abierto:
    En la teoría de conjuntos Von Neumann–Bernays–Gödel (NBG) sin elección, se pueden definir los cardinales de Reinhardt como $\crit{j}$
    donde $j\colon V\prec V$. Para intentar avanzar en la pregunta anterior,
    Golberg \cite{goldberg_reinhardt_2021} propone en el contexto de NBG la noción de cardinal débilmente Reinhardt.
    
    \section{Objetivos}
    \begin{itemize}
        \item Realizar un estudio comparativo las tres demostraciones del teorema \ref{eq:kunen_inc} mencionadas antes,
        haciendo énfasis en las fortalezas y debilidades de cada demostración.
        \item Estudiar la relación entre el teorema de inconsistencia y el problema abierto relacionado.
        \item Estudiar el estado actual del desarrollo del problema, dando un breve resumen de los resultados recientes. 
    \end{itemize}

    \nocite{jech_set_2003}
    \nocite{kanamori_mathematical_1996}
    \singlespacing
    \sloppy
    \printbibliography
\end{document}
