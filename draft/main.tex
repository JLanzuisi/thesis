\documentclass[12pt]{article}

\usepackage{base}
\usepackage{draft}

\usepackage{lipsum}

\begin{document}
    \maketitle
    \thispagestyle{empty}

    \begin{abstract}
        A continuación se presentan los objetivos, metodología, justificación,
        cronograma e introducción en el marco del proyecto de grado de el
        Br. Jhonny Alexander Lanzuisi Berrizbeitia, Carnet No. 15-10759, para su
        evaluación y posible aprobación por la Coordinación de Matemáticas de
        la USB. Cabe señalar que las posibles reformas que la misma dinámica
        de la investigación impongan o se susciten durante el desarrollo de
        este proyecto también serían presentadas oportunamente.
    \end{abstract}

    Math test: $ x = y $ and $ a^2 - b^2 = c^2 $.

    \section{Fases del proyecto}

    \begin{enumerate}[label=\Roman*.]
        \item Búsqueda y recolección de datos mediante la revisión
              bibliográfica y el establecimiento de conjeturas y conclusiones.
        \item Redacción del trabajo de grado.
    \end{enumerate}

    \section{Cronograma}

    \begin{description}
        \let\olditem\item
        \renewcommand\item[1][]{\olditem[\normalfont #1.]}
        \item[1er trimestre]
            \begin{minipage}[t]{\dimexpr\linewidth-2cm\relax}
                \begin{description}
                    \item[Semanas 1-4] Búsqueda y recolección de la bibliografía.
                    \item[Semanas 5-8] Revisión de la bibliografía planteada.
                    \item[Semanas 9-12] Redacción de la introducción y el primer capítulo
                                        del trabajo de grado.
                \end{description}
            \end{minipage}
        \item[2er trimestre]
            \begin{minipage}[t]{\linewidth}
                \begin{description}
                    \item[Semanas 1-12] Redacción de los capítulos restantes del trabajo de grado
                \end{description}
            \end{minipage}
        \item[3er trimestre]
            \begin{minipage}[t]{\linewidth}
                \begin{description}
                    \item[Semanas 1-6] Revisión del borrador y redacción final del
                                       proyecto de grado.
                    \item[Semanas 7-12] Presentación y defensa.
                \end{description}
            \end{minipage}
    \end{description}

    \section{Justificación}
    Algo: \cite{kanamori_higher_2009} heeeey
    \section{Modalidad y metodologia}
    \section{Introduccion}
    \singlespacing
    \printbibliography
\end{document}
